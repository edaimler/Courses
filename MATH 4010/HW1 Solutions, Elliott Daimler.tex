%This is my super simple Real Analysis Homework template

\documentclass{article}
\usepackage[utf8]{inputenc}
\usepackage[english]{babel}
\usepackage[]{amsthm} %lets us use \begin{proof}
\usepackage[]{amssymb} %gives us the character \varnothing
\usepackage{fancyhdr}
\usepackage{amsmath}
\usepackage[margin=1in]{geometry} %reduce margins to 0.5 inch

\setlength{\parindent}{0pt}
\setlength{\parskip}{1em}

%\pagestyle{fancy}
\makeatletter
\renewcommand*\env@matrix[1][*\c@MaxMatrixCols c]{%
  \hskip -\arraycolsep
  \let\@ifnextchar\new@ifnextchar
  \array{#1}}
\makeatother

\renewcommand*\descriptionlabel[1]{\hspace\labelsep\normalfont #1}

\title{\textbf{Homework 1 Solutions}}
\date{}
%This information doesn't actually show up on your document unless you use the maketitle command below

\fancypagestyle{firstpage}{%
  \rhead{
	Elliott Daimler \\
	MATH 4010 \\
	Dr. Chastkofsky \\ 
	MWF 10:20am \\
	\date\today
  }
  
}


\begin{document}
\maketitle %This command prints the title based on information entered above

\thispagestyle{firstpage}
%Section and subsection automatically number unless you put the asterisk next to them.
\section*{1}

Claim: Let $G$ be a finite Abelian group with identity $e$ and $a_1, a_2, \cdots, a_n$ in $G$, and let 
$x = a_1 a_2 \cdots a_n$.  Then $x^2 = e$.

\begin{proof}
    Let some $a_i$ in $G$ be given.  Since $G$ is finite, there is some $k_i$ in $\mathbb{Z}^+$
    such that $a_i^{k_i} = e$.  Consider the values of $k_i$.
    \begin{description}
        \item[Case 1: $k_i = 1$.]  Then $a_i$ is the identity.
        \item[Case 2: $k_i = 2$.]  Since $G$ is Abelian, we have
    \end{description}
    \begin{alignat*}{1}
        x^2 &= (a_1 a_2 \cdots a_i \cdots a_n)^2 \\
        &= a_i^2(a_1 a_2 \cdots a_n)^2 \\
        &= e(a_1 a_2 \cdots a_n)^2,
    \end{alignat*}
    assuming without loss of generality that $2 < i < n$.  
    The remaining factors of $x^2$ are considered in case 3.
    \begin{description}
        \item[Case 3: $k_i > 2$.]  Then there is some $a_i^{-1} \neq a_i$ in $G$ such that 
        $a_i^{-1} = a_i^{k-1}$.  Since inverses are unique, $a_i$ and 
        $a_i^{-1}$ are unique inverses of one another.  $G$ is Abelian, so 
        every $a_i$ and $a_i^{-1}$ in $x$ produce the identity.
    \end{description}
\end{proof}

\section*{2}

Claim: Let $G$ be a group such that for every $a, b, c$ in $G$ where $ab=ca$ then 
$b=c$.  Then $G$ is Abelian.

\begin{proof} 
    Let $a, b$ in $G$ be given, and consider the product $aba^{-1}$.  By the closure 
    property, there is some $c$ in $G$ such that $aba^{-1} = c$.  Multiplying  
    both terms of this equation by $a$ on the right side yields 
    \begin{alignat*}{1}
        aba^{-1} a &= c a\\ 
        \rightarrow a b e &= c a \\ 
        \rightarrow a b &= c a,
    \end{alignat*}
    so $b = c$, and $ab = ba$.
\end{proof}

\section*{3}

Claim: Let $G$ be a group where every nonidentity element $a$ in $G$ has order $2$.  Then 
$G$ is Abelian.

\begin{proof} 
    Let elements $a, b$ of $G$ be given.  Then for the product $ab$, we have 
    \begin{alignat*}{1}
        ab(ab)^{-1} &= ab(b^{-1}a^{-1}) \\
        &= e.
    \end{alignat*}
    Since $a^2 = e$ and $b^2 = e$, we have $a^{-1} = a$ and $b^{-1} = b$ by uniqueness 
    of inverses, so 
    \begin{alignat*}{1}
        ab(b^{-1}a^{-1}) &= e \\
        \rightarrow ab(ba) &= e \\
        \rightarrow ab(ba) &= (ab)^2.
    \end{alignat*}
    By the cancellation property, the equation $ab(ba) = ab(ab)$ implies $ba = ab$.
\end{proof}

\end{document}