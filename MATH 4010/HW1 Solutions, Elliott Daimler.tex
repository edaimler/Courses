%This is my super simple Real Analysis Homework template

\documentclass{article}
\usepackage[utf8]{inputenc}
\usepackage[english]{babel}
\usepackage[]{amsthm} %lets us use \begin{proof}
\usepackage[]{amssymb} %gives us the character \varnothing
\usepackage{fancyhdr}
\usepackage{amsmath}
\usepackage[margin=1in]{geometry} %reduce margins to 0.5 inch

\setlength{\parindent}{0pt}
\setlength{\parskip}{1em}

%\pagestyle{fancy}
\makeatletter
\renewcommand*\env@matrix[1][*\c@MaxMatrixCols c]{%
  \hskip -\arraycolsep
  \let\@ifnextchar\new@ifnextchar
  \array{#1}}
\makeatother

\title{\textbf{Homework 1 Solutions}}
\date{}
%This information doesn't actually show up on your document unless you use the maketitle command below

\fancypagestyle{firstpage}{%
  \rhead{
	Elliott Daimler \\
	MATH 4010 \\
	Dr. Chastkofsky \\ 
	MWF 10:20am \\
	\date\today
  }
  
}


\begin{document}
\maketitle %This command prints the title based on information entered above

\thispagestyle{firstpage}
%Section and subsection automatically number unless you put the asterisk next to them.
\section*{1}

\subsection*{(a)}
 
To evaluate $(x+1)^3$ in $\mathbb{Z}_3 [x]$, we'll first expand the polynomial:

\begin{alignat*}{1}
    (x+1)^3 &= x^3 + 3x^2 + 3x + 1 \\ 
\end{alignat*}

Now we evaluate in $\mathbb{Z}_3$.  In $\mathbb{Z}_3$, each coefficient in the 
polynomial is congruent to some integer $a$ for $0 \leq a < 3$, so we have 
$x^3 + 3x^2 + 3x + 1 = x^3 + 1$ mod $3$.


\subsection*{(b)}

Similar to above, to evaluate $(x-1)^5$ in $\mathbb{Z}_5 [x]$ we'll expand the 
polynomial: 

\begin{alignat*}{1}
    (x-1)^5 = x^5 - 5x^4 + 10x^3 - 10x^2 + 5x -1 \\ 
\end{alignat*}

In $\mathbb{Z}_5$, each coefficient in the polynomial is congruent to some integer 
$a$ for $0 \leq a < 5$, so we have $x^5 - 5x^4 + 10x^3 - 10x^2 + 5x -1 = x^5 - 1$ mod $5$.

\section*{2}

\subsection*{(a)} 

Given $f(x) = x^4 - 7x + 1$ and $g(x) = 2x^2 + 1$ in $\mathbb{Q}[x]$, we can choose 
$q(x) = (\frac{1}{2}x^2 - \frac{1}{4})$ and $r(x) = 7x + \frac{5}{4}$.  The degree of 
$r(x)$ is nonzero and less than the degree of $f(x)$.  Furthermore,  

\begin{alignat*}{1}
    \bigg(2x^2 + 1\bigg)\bigg(\frac{1}{2}x^2 - \frac{1}{4}\bigg) + \bigg(7x + \frac{5}{4}\bigg) &= x^4 - 7x + 1. \\ 
\end{alignat*}

The left side of the above equation is $g(x)q(x) + r(x)$ and the right side is 
$f(x)$, so we have $g(x)q(x) + r(x) = f(x)$.

\subsection*{(b)} 

Given $f(x) = 4x^4 + 2x^3 + 6x^2 + 4x + 5$ and and $g(x) = 3x^2 + 2$ in $\mathbb{Z}_7 [x]$, we can choose 
$q(x) = (6x^2 + x)$ and $r(x) = -x^3 + x^2 + 2x + 5$.  The degree of 
$r(x)$ is nonzero and less than the degree of $f(x)$.  Furthermore, 

\begin{alignat*}{1}
    (3x^2 + 2)(6x^2 + x) + -x^3 + x^2 + 2x + 5 &= 18x^4 + 3x^3 + 12x^2 + 2x + (-x^3 + x^2 + 2x + 5) \textrm{ mod } 7 \\ 
    &= 4x^4 + 3x^3 5x^2 + 2x + (-x^3 + x^2 + 2x + 5) \textrm{ mod } 7. \\ 
    &= 4x^4 + 2x^3 + 6x^2 + 4x + 5
\end{alignat*}

The left side of the above equation is $g(x)q(x) + r(x)$ and the right side is 
$f(x)$, so we have $g(x)q(x) + r(x) = f(x)$.



\section*{3}

For the polynomial $3x + 1$, we can multiply by $6x+1$ to get the multiplicative 
identity in $\mathbb{Z}_9[x]$:

\begin{alignat*}{1}
    (3x+1)(6x+1) &= 18x + 9x + 1 \\
        &= 1 \textrm{ mod } 9.
\end{alignat*}

The resulting polynomial above is not a constant polynomial, but this does not contradict 
Corrolary 4.5 because $\mathbb{Z}_9[x]$ is not an integral domain.

\section*{4}

The map $D: \mathbb{R}[x] \rightarrow \mathbb{R}[x]$ is not a ring homomorphism, because 
it does not preserve multiplication or the multiplicative identity.  Consider the values 
$(x + 1)$ and $(x + 2)$.  Then 

\begin{alignat*}{1}
    D[(x+1)(x+2)] &= D(x^2 + 3x + 2) \\ 
    &= 2x + 3,
\end{alignat*}

and alternatively, 

\begin{alignat*}{1}
    D(x+1)D(x+2) &= 1 \cdot 1 \\ 
    &= 1. \\
\end{alignat*}

Now consider an identity element of the particular ring of polynomials, say $a_0$.  
Then 

\begin{alignat*}{1}
    D(a_0x) &= a_0 \cdot 1 \\
    &= a_0
\end{alignat*}

and alternatively, 

\begin{alignat*}{1}
    D(a_0)D(x) &= 0 \cdot 1 \\ 
    &= 0.
\end{alignat*}

The counterexamples above show that the map is not a ring homomorphism.

\section*{5}

Suppose there is some polynomial $f[x]$ in $F[x]$ that divides $x + a$ and $x + b$.  
Then $f(x)$ must have a degree less than $x + a$ and $x + b$; these two expressions are 
both degree one, so $f(x)$ is degree zero, meaning it is the multiplicative identity.  So 
$x + a$ and $x + b$ are relatively prime. 


\section*{6}

\subsection*{(a)}

For the polynomial $x^4 - x^3 - x^2 + 1$, an easy root is $1$, so we can factor out $(x - 1)$.  
This gives us $(x^4 - x^3 - x^2 + 1)/(x-1) = (x^3 - x - 1)$.  The result has no more factors in 
$\mathbb{Q}[x]$, so we are done.
For the polynomial $x^3 - 1$, an easy root is again $1$, so we can factor out $(x - 1)$.  
This gives us $(x^3 - 1)/(x - 1) = (x^2 + x + 1)$.  The result has no more factors in 
$\mathbb{Q}[x]$, so we are done.  The gcd for both polynomials is $(x - 1)$.

\subsection*{(b)}

For the polynomial $x^4 + 3x^3 + 2x + 4$, one root is $-1$, so we can factor out $(x + 1)$.
Then we have $(x^4 + 3x^3 + 2x + 4)/(x + 1) = (x^3 + 2x^2 - 2x + 4)$.  The result has 
no more factors in $\mathbb{Z}_5[x]$, so we are done.
For the polynomial $x^2 - 1$, one root is $1$, so we can factor out $(x - 1)$.
Then we have $(x^2 - 1)/(x + 1) = (x - 1)$.  The result has no more factors in 
$\mathbb{Z}_5[x]$, so we are done.

\section*{7}

For the factor $(x - 1)$ in part (a) above, we compare the coefficients in the equation

\begin{alignat*}{1}
    x - 1 = a(x^4 - x^3 - x^2 + 1) + b(x^3 - 1)
\end{alignat*}

which gives $2a - 2b = -2$ for $x = -1$, $a - b = -1$ for $x = 0$, and 
$5a + 7b = 1$ for $x = 2$.  The solution is $a = -1/2$, $b = 1/2$ and 

\begin{alignat*}{1}
    -\frac{1}{2}(x^4 - x^3 - x^2 + 1) + \frac{1}{2}(x^3 - 1) = (x - 1). \\
\end{alignat*}

For the factor $(x + 1)$ in part (b) above, we compare the coefficients in the equation 

\begin{alignat*}{1}
    x + 1 = a(x^4 + 3x^3 + 2x + 4) + b(x^2 - 1) 
\end{alignat*}

which gives $4a - b = 1$ for $x = 0$, and $10a = 2$ for $x = 1$.  The solution is 
$a = 1/5$, $b = -1/5$ and 

\begin{alignat*}{1}
    \frac{1}{5}(x^4 + 3x^3 + 2x + 4) - \frac{1}{5}(x^2 - 1) = x + 1. \\ 
\end{alignat*}

\end{document}