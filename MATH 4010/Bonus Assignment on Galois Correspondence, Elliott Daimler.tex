%This is my super simple Real Analysis Homework template

\documentclass{article}
\usepackage[utf8]{inputenc}
\usepackage[english]{babel}
\usepackage[]{amsthm} %lets us use \begin{proof}
\usepackage[]{amssymb} %gives us the character \varnothing
\usepackage{fancyhdr}
\usepackage{amsmath}
\usepackage[margin=1in]{geometry} %reduce margins to 0.5 inch

\setlength{\parindent}{0pt}
\setlength{\parskip}{1em}

%\pagestyle{fancy}
\makeatletter
\renewcommand*\env@matrix[1][*\c@MaxMatrixCols c]{%
  \hskip -\arraycolsep
  \let\@ifnextchar\new@ifnextchar
  \array{#1}}
\makeatother

\renewcommand*\descriptionlabel[1]{\hspace\labelsep\normalfont #1}

\title{\textbf{Bonus Assignment, Galois Correspondence Solutions}}
\date{}
%This information doesn't actually show up on your document unless you use the maketitle command below

\fancypagestyle{firstpage}{%
  \rhead{
	Elliott Daimler \\
	MATH 4010 \\
	Dr. Chastkofsky \\ 
	MWF 10:20am \\
	\date\today
  }
  
}


\begin{document}
\maketitle %This command prints the title based on information entered above

\thispagestyle{firstpage}
%Section and subsection automatically number unless you put the asterisk next to them.

\section*{1}

Claim: Let $f(x) = x^3 - 2$ in $\mathbb{Q}[x]$, and let $\omega = e^{2 \pi i /3}$.  Then 
for $\sigma$, $\tau \in \mathrm{Gal}(\mathbb{Q}[2^{1/3}, \omega 2^{1/3}, \omega^2 2^{1/3}] / \mathbb{Q})$ 
where $\sigma: 2^{1/3} \rightarrow \omega 2^{1/3}$, $\omega \rightarrow \omega$ and 
$\tau: 2^{1/3} \rightarrow 2^{1/3}$, $\omega \rightarrow \omega^2$, the fixed 
field of $< \sigma \tau >$ is $\mathbb{Q}[\omega ^2 2^{1/3}]$ and that of 
$< \sigma ^2 \tau >$ is $\mathbb{Q}[\omega 2^{1/3}]$.

\begin{proof}
  For the subgroup $< \sigma \tau >$, we have 
  \begin{alignat*}{1} 
    \sigma \tau (\omega ^2 2^{1/3}) &= \sigma (\omega ^4 2^{1/3}) \\ 
    &= \sigma (\omega \omega ^3 2^{1/3}) \\ 
    &= \sigma (\omega 2^{1/3}) \\ 
    &= \omega ^2 2^{1/3},
  \end{alignat*}

  The base field is $\mathbb{Q}$, so $\sigma \tau$ acts as an identity permutation on 
  $\mathbb{Q}[\omega^2 2^{1/3}]$.  The subgroup is generated by $\sigma \tau$ so 
  each $a \in < \sigma \tau >$ is of the form $a = (\sigma \tau)^k$ for some 
  $k \in \mathbb{Z}$, and $a$ acts as a product of identity permutations on 
  $\mathbb{Q}[\omega^2 2^{1/3}]$ and thus fixes the field. \\ 
  \\ 
  For the subgroup $< \sigma^2 \tau >$, we have 
  \begin{alignat*}{1} 
    \sigma^2 \tau (\omega 2^{1/3}) &= \sigma^2 (\omega^2 2^{1/3}) \\ 
    &= \sigma (\omega^2 \omega 2^{1/3}) \\ 
    &= \omega ^2 \omega \omega 2^{1/3} \\ 
    &= \omega ^3 \omega 2^{1/3} \\ 
    &= \omega 2^{1/3}.
  \end{alignat*}

Again, the base field is $\mathbb{Q}$, and $\sigma^2 \tau$ acts as an identity 
permutation on $\mathbb{Q}[\omega 2^{1/3}]$.  Using an argument similar to that 
used for $<\sigma \tau>$, we have that the subgroup $<\sigma^2 \tau>$ fixes
$\mathbb{Q}[\omega 2^{1/3}]$.
\end{proof}
\newpage
\section*{2}

Let $f(x) = x^6 + x^5 + x^4 + x^3 + x^2 + x + 1$ and let $\omega = e^{2 \pi i/7}$ 
be a 7th root of unity.  The polynomial $x^7 - 1$ has $\omega$ as a root, 
so must have $x-1$ as a factor.  We then have 
\begin{alignat*}{1}
  \frac{x^7 - 1}{x - 1} &= x^6 + x^5 + x^4 + x^3 + x^2 + x + 1.
\end{alignat*}
Let $a = \omega^6 + \omega$.  Then $a^2 = \omega^5 + \omega^2 + 2$ and 
$a^3 = \omega^4 + \omega^3 + 3 \omega^6 + 3 \omega$.  We can rewrite this 
into the form $a^3 + a^2 - 2a - 1 = 0$, which has no rational roots in $\mathbb{Q}$, 
so is not reducible in $\mathbb{Q}$.  So the minimal polynomial has degree 6 
and must be the above polynomial.\\
\\ 
The degree of $f(x)$ is 6, so it must have 6 roots.  
Define $\sigma \in \mathrm{Gal}\mathbb{Q}[\omega]$ as $\sigma(\omega) = \omega^3$.  Then 
$\sigma (\omega ) = \omega^3$; $\sigma (\omega^3 ) = \omega^9 = \omega^2$; 
$\sigma (\omega^2 ) = \omega^6$; $\sigma (\omega^6 ) = \omega^{18} = \omega^4$; 
$\sigma (\omega^4 ) = \omega^{12} = \omega^5$; $\sigma (\omega^5 ) = \omega^{15} = \omega$.  
So $\sigma$ generates all roots of the polynomial. \\ 
\\ 
We also have $\sigma^3(\omega) = \omega^{27} = \omega^6$ and 
$\sigma^3(\omega^6) = \sigma^2(\omega^4) = \sigma(\omega^5) = \omega$, so 
$< \sigma^3 >$ is a proper subgroup of order 2 of $< \sigma >$, and 
$\sigma^2(\omega) = \omega^9 = \omega^2$, $\sigma^2(\omega^2) = \omega^4$, 
and $\sigma^2(\omega^4) = \omega$ so $< \sigma^2 >$ is a proper subgroup 
of order 3 of $< \sigma >$. \\ 
\\ 
We permute $\omega^6 + \omega$ by $\sigma^3$ to get $\sigma^3(\omega^6 + \omega) = 
\sigma^3(\omega^6) + \sigma^3(\omega) = \omega + \omega^6$.  $\mathbb{Q}$ is 
the base field and is fixed by $\sigma$, so $\mathbb{Q}[\omega^6 + \omega]$ 
is fixed by $\sigma^3$. \\ 
\\ 
Similarly, we permute $\omega^4 + \omega^2 + \omega$ by $\sigma^2$ to get 
$\sigma^2(\omega^4 + \omega^2 + \omega) = \sigma^2(\omega^4) + 
\sigma^2(\omega^2) + \sigma^2(\omega) = \omega + \omega^4 + \omega^2$.  
Again, $\sigma$ fixes the base field $\mathbb{Q}$, so 
$\mathbb{Q}[ \omega^4 + \omega^2 + \omega]$ is fixed by $\sigma^2$.

\end{document}