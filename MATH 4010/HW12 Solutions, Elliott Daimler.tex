%This is my super simple Real Analysis Homework template

\documentclass{article}
\usepackage[utf8]{inputenc}
\usepackage[english]{babel}
\usepackage[]{amsthm} %lets us use \begin{proof}
\usepackage[]{amssymb} %gives us the character \varnothing
\usepackage{fancyhdr}
\usepackage{amsmath}
\usepackage[margin=1in]{geometry} %reduce margins to 0.5 inch

\setlength{\parindent}{0pt}
\setlength{\parskip}{1em}

%\pagestyle{fancy}
\makeatletter
\renewcommand*\env@matrix[1][*\c@MaxMatrixCols c]{%
  \hskip -\arraycolsep
  \let\@ifnextchar\new@ifnextchar
  \array{#1}}
\makeatother

\renewcommand*\descriptionlabel[1]{\hspace\labelsep\normalfont #1}

\title{\textbf{Homework 12, Fundamental Theorem Solutions}}
\date{}
%This information doesn't actually show up on your document unless you use the maketitle command below

\fancypagestyle{firstpage}{%
  \rhead{
	Elliott Daimler \\
	MATH 4010 \\
	Dr. Chastkofsky \\ 
	MWF 10:20am \\
	\date\today
  }
  
}


\begin{document}
\maketitle %This command prints the title based on information entered above

\thispagestyle{firstpage}
%Section and subsection automatically number unless you put the asterisk next to them.

Claim: Let $K/F$ be Galois, let $G = \mathrm{Gal}(K/F)$, let $E$ be an intermediate 
field, and let $H = \mathrm{Gal}(K/E)$ so that $E = \mathrm{Inv}(H)$.  
Then $E/F$ is Galois if and only if $\mathrm{Inv}(E)$ is a normal subgroup.

\begin{proof} 
    We'll prove the claim in three sections.
    \section*{1}
    
    Let $H$ be any subgroup of $G$, and define $g(E) = \{ ga | g \in G, a \in E \}$ and 
    $gHg^{-1} = \{ ghg^{-1} | g \in G, h \in H \}$.  Then 
    for $b = ga$, $ghg^{-1}b = ghg^{-1}(ga) = gha$.  Since $a$ is fixed by 
    $h$, $gha = ga = b$, so $ghg^{-1}b = b$ for any $b \in g(E)$. \\ 
    \\
    Conversely, let some $b \in \mathrm{Inv}(gHg^{-1})$ be given.
    There is some $a \in K$ such that $g^{-1}b = a$, so $b = ga$ and 
    $ghg^{-1}b = ghg^{-1}ga = gha = b$.  But $ga = b$, so $ga = gha$ and 
    $a$ must therefore be fixed by $h$.  So $a$ is in $E$ and $b \in g(E)$.  
    It follows that $\mathrm{Inv}(gHg^{-1}) = g(E)$.

    \section*{2} 

    Assume that $g(E) = E$ for all $g \in G$.  Let $f_g(j) = gj$ 
    for $j \in \mathrm{Gal}(E/F)$, and define a mapping 
    $\varphi : \mathrm{Gal}(K/F) \rightarrow \mathrm{Gal}(E/F)$ by 
    $\varphi(g) = f_g(j)$  For $g \in \mathrm{Gal}(K/E)$, $g$ fixes 
    elements of $E$ so $f_g(j) = j = f_{id}(j)$ where $id$ is the identity 
    permutation.  This implies that $\mathrm{Gal}(K/E)$ is the kernel of 
    $\varphi$ and $j \in \mathrm{Gal}(E/F)$ are mapped to non-identity 
    permutations of $E$. \\ 
    \\ 
    Now let $a, b \in G$ be given.  Then $\varphi(ab) = f_{ab}(j) = abj$. 
    Similarly, $\varphi(a)\varphi(b) = f_a \circ f_b(j) = f_a(bj) = abj$, so 
    $\varphi$ is a homomorphism of the Galois group.  

    \section*{3} 

    Suppose that $E$ is Galois over $F$; then $E$ is the splitting field of 
    some $f(x) \in F[x]$ and all roots of $f(x)$ are contained in $E$.  
    For any $g \in G$, $g$ permutes roots of $f(x)$ by Theorem 12.2 of 
    Hungerford.  Since all roots of $f(x)$ are in $E$, it follows that 
    for any $e \in E$, $ge \in E$ and thus $g(E) = E$.  \\ 
    \\ 
    We have shown that 
    $\mathrm{Inv}(E)$ being normal in $G$ implies that $E$ is Galois 
    over $F$, and conversely, if $E$ is Galois over $F$, then $g(E) 
    = E = gHg^{-1}$ and $\mathrm{Inv}(E)$ is normal in $G$.



\end{proof}

\end{document}