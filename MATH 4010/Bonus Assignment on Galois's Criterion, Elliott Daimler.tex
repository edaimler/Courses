%This is my super simple Real Analysis Homework template

\documentclass{article}
\usepackage[utf8]{inputenc}
\usepackage[english]{babel}
\usepackage[]{amsthm} %lets us use \begin{proof}
\usepackage[]{amssymb} %gives us the character \varnothing
\usepackage{fancyhdr}
\usepackage{amsmath}
\usepackage[margin=1in]{geometry} %reduce margins to 0.5 inch

\setlength{\parindent}{0pt}
\setlength{\parskip}{1em}

%\pagestyle{fancy}
\makeatletter
\renewcommand*\env@matrix[1][*\c@MaxMatrixCols c]{%
  \hskip -\arraycolsep
  \let\@ifnextchar\new@ifnextchar
  \array{#1}}
\makeatother

\renewcommand*\descriptionlabel[1]{\hspace\labelsep\normalfont #1}

\title{\textbf{Bonus Assignment, Galois's Criterion Solutions}}
\date{}
%This information doesn't actually show up on your document unless you use the maketitle command below

\fancypagestyle{firstpage}{%
  \rhead{
	Elliott Daimler \\
	MATH 4010 \\
	Dr. Chastkofsky \\ 
	MWF 10:20am \\
	\date\today
  }
  
}


\begin{document}
\maketitle %This command prints the title based on information entered above

\thispagestyle{firstpage}
%Section and subsection automatically number unless you put the asterisk next to them.

\section*{1}

Let $f(x) = x^5 - 5x^3 + 5x + 2$ and $\alpha = \omega^9 + \omega$ where $\omega = e^{2 \pi i/10}$. 
The polynomial can be factored into the form $f(x) = (x+2)(x^2 - x - 1)^2$, 
has a root $-2$, and has two repeating (irrational) real roots for $(x^2 - x - 1)$, 
so all 5 roots are real.  
The trigonometric form of $\alpha$ is 
\begin{alignat*}{1}
  \alpha &= e^{18 \pi i/10} + e^{2 \pi i/10} \\
  &= \mathrm{cos}\bigg( \frac{18 \pi}{10} \bigg) + i \mathrm{sin}\bigg( \frac{18 \pi}{10} \bigg) + 
  \mathrm{cos}\bigg( \frac{2 \pi}{10} \bigg) + i \mathrm{sin}\bigg( \frac{2 \pi}{10} \bigg).\\ 
\end{alignat*}

Using the sum-to-product identities of sine and cosine, this becomes 

\begin{alignat*}{1}
  \mathrm{cos}\bigg( \frac{18 \pi}{10} \bigg) + i \mathrm{sin}\bigg( \frac{18 \pi}{10} \bigg) + 
  \mathrm{cos}\bigg( \frac{2 \pi}{10} \bigg) + i \mathrm{sin}\bigg( \frac{2 \pi}{10} \bigg) 
  &= 2 \mathrm{cos}\bigg( \frac{20 \pi}{20} \bigg)\mathrm{cos}\bigg( \frac{16 \pi}{20} \bigg)
  + 2i \mathrm{sin}\bigg( \frac{20 \pi}{20} \bigg)\mathrm{sin}\bigg( \frac{16 \pi}{20} \bigg) \\
  &= -2 \mathrm{cos} \bigg( \frac{4 \pi}{5} \bigg) + 0 \\ 
  &= -2 \mathrm{cos} \bigg( \frac{4 \pi}{5} \bigg). \\ 
\end{alignat*}

The values of $\alpha ^3$ and $\alpha^5$ are then 

\begin{alignat*}{1}
  \alpha^3 &= \bigg[ -2 \mathrm{cos} \bigg( \frac{4 \pi}{5} \bigg) \bigg]^3 \\ 
  &= -4 \mathrm{cos} \bigg( \frac{4 \pi}{5} \bigg) - 4\mathrm{cos} \bigg( \frac{8 \pi}{5} \bigg) \mathrm{cos} \bigg( \frac{4 \pi}{5} \bigg) \\ 
  &= -4 \mathrm{cos} \bigg( \frac{4 \pi}{5} \bigg) - 2 \mathrm{cos} \bigg( \frac{4 \pi}{5} \bigg) - 2 \mathrm{cos} \bigg( \frac{12 \pi}{5} \bigg), \\ 
  \\ 
  \alpha ^5 &= \bigg[ -2 \mathrm{cos} \bigg( \frac{4 \pi}{5} \bigg) \bigg]^5 \\ 
  &= -12 \mathrm{cos} \bigg( \frac{4 \pi}{5} \bigg) -8\mathrm{cos} \bigg( \frac{4 \pi}{5} \bigg) -8\mathrm{cos} \bigg( \frac{12 \pi}{5} \bigg)
  - 2 \mathrm{cos} \bigg( \frac{12 \pi}{5} \bigg) - 2 \\ 
  &= -20 \mathrm{cos} \bigg( \frac{4 \pi}{5} \bigg) - 10 \mathrm{cos} \bigg( \frac{12 \pi}{5} \bigg) - 2,
\end{alignat*}

and $f(\alpha)$ becomes 

\begin{alignat*}{1}
  f(\alpha) &= -20 \mathrm{cos} \bigg( \frac{4 \pi}{5} \bigg) - 10 \mathrm{cos} \bigg( \frac{12 \pi}{5} \bigg) - 2 
  -5 \bigg[ -4 \mathrm{cos} \bigg( \frac{4 \pi}{5} \bigg) - 2 \mathrm{cos} \bigg( \frac{4 \pi}{5} \bigg) - 2 \mathrm{cos} \bigg( \frac{12 \pi}{5} \bigg) \bigg] 
  + 5\bigg[ -2 \mathrm{cos} \bigg( \frac{4 \pi}{5} \bigg) \bigg] \\ 
  &= -20 \mathrm{cos}\bigg( \frac{4 \pi}{5} \bigg) -10 \mathrm{cos}\bigg( \frac{12 \pi}{5} \bigg) 
  - 2 + 20 \mathrm{cos}\bigg( \frac{4 \pi}{5} \bigg) + 10 \mathrm{cos}\bigg( \frac{4 \pi}{5} \bigg) 
  + 10 \mathrm{cos}\bigg( \frac{12 \pi}{5} \bigg) - 10 \mathrm{cos}\bigg( \frac{4 \pi}{5} \bigg) + 2 \\ 
  &= 0.  
\end{alignat*}

So $\alpha$ is a root of $f(x)$.  Since $\alpha$ is real and 
all 5 roots are real, $\mathbb{Q}[\alpha]$ 
splits $f(x)$.

\end{document}