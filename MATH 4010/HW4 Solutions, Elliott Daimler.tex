%This is my super simple Real Analysis Homework template

\documentclass{article}
\usepackage[utf8]{inputenc}
\usepackage[english]{babel}
\usepackage[]{amsthm} %lets us use \begin{proof}
\usepackage[]{amssymb} %gives us the character \varnothing
\usepackage{fancyhdr}
\usepackage{amsmath}
\usepackage[margin=1in]{geometry} %reduce margins to 0.5 inch

\setlength{\parindent}{0pt}
\setlength{\parskip}{1em}

%\pagestyle{fancy}
\makeatletter
\renewcommand*\env@matrix[1][*\c@MaxMatrixCols c]{%
  \hskip -\arraycolsep
  \let\@ifnextchar\new@ifnextchar
  \array{#1}}
\makeatother

\renewcommand*\descriptionlabel[1]{\hspace\labelsep\normalfont #1}

\title{\textbf{Homework 5, Section 8.2 Solutions}}
\date{}
%This information doesn't actually show up on your document unless you use the maketitle command below

\fancypagestyle{firstpage}{%
  \rhead{
	Elliott Daimler \\
	MATH 4010 \\
	Dr. Chastkofsky \\ 
	MWF 10:20am \\
	\date\today
  }
  
}


\begin{document}
\maketitle %This command prints the title based on information entered above

\thispagestyle{firstpage}
%Section and subsection automatically number unless you put the asterisk next to them.
\section*{1}

Claim: Suppose that $N$ is a subgroup of $G$ and that $N$ has the property that if $a$ and $b$ are any elements of $G$, then $ab$ belongs 
to $N$ if and only if $ba$ belongs to $N$.  Then $N$ is a normal subgroup.





\section*{2}

Claim: The inner automorphisms of a group $G$ are a normal subgroup of $\mathrm{Aut}(G)$. 

\begin{proof}
  Since the inner automorphisms of $G$ form a subgroup, then the inverse of each automorphism $\varphi$ in $\mathrm{Inn}(G)$ 
  is also in the subgroup.
  \begin{alignat*}{1}
    \varphi_g \circ \varphi_x \circ \varphi_g ^{-1} \circ \varphi_h^{-1} = gxg^{-1}hxh^{-1}gx^{-1}g^{-1}
  \end{alignat*}
\end{proof}

\section*{3}

Claim: If $K$ is a characteristic subgroup of $N$ and $N$ is a normal subgroup of $G$, then $K$ is a normal subgroup of 
$G$. 


\section*{4} 

Claim: Suppose that $G$ is a group all of whose subroups are normal.  If $a$, $b$, are elements of $G$, then $ab = ba^k$ for 
some integer $k$.




\end{document}