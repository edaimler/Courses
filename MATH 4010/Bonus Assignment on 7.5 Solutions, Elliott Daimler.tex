%This is my super simple Real Analysis Homework template

\documentclass{article}
\usepackage[utf8]{inputenc}
\usepackage[english]{babel}
\usepackage[]{amsthm} %lets us use \begin{proof}
\usepackage[]{amssymb} %gives us the character \varnothing
\usepackage{fancyhdr}
\usepackage{amsmath}
\usepackage[margin=1in]{geometry} %reduce margins to 0.5 inch

\setlength{\parindent}{0pt}
\setlength{\parskip}{1em}

%\pagestyle{fancy}
\makeatletter
\renewcommand*\env@matrix[1][*\c@MaxMatrixCols c]{%
  \hskip -\arraycolsep
  \let\@ifnextchar\new@ifnextchar
  \array{#1}}
\makeatother

\renewcommand*\descriptionlabel[1]{\hspace\labelsep\normalfont #1}

\title{\textbf{Bonus Assignment on 7.5 Solutions}}
\date{}
%This information doesn't actually show up on your document unless you use the maketitle command below

\fancypagestyle{firstpage}{%
  \rhead{
	Elliott Daimler \\
	MATH 4010 \\
	Dr. Chastkofsky \\ 
	MWF 10:20am \\
	\date\today
  }
  
}


\begin{document}
\maketitle %This command prints the title based on information entered above

\thispagestyle{firstpage}
%Section and subsection automatically number unless you put the asterisk next to them.
\section*{1}
Claim: If $G = S_3$ then $\mathrm{Inn}(G) = \mathrm{Aut}(G)$. 

\begin{proof}
    We know that $\mathrm{Inn}(G) \subseteq \mathrm{Aut}(G)$, so we must show that $\mathrm{Aut}(G) \subseteq \mathrm{Inn}(G)$.  Since $S_3$ has only 6 elements, we will list the elements explicitly: \\

    $(1 \: 2 \: 3) = (1 \: 2)(1 \: 3 \: 2)(1 \: 2)$ \\
    $(1 \: 3 \: 2) = (1 \: 2)(1 \: 2 \: 3)(1 \: 2)$ \\
    $(1 \: 2) = (1 \: 2)(1 \: 2)(1 \: 2)$ \\
    $(1 \: 3) = (1 \: 2)(2 \: 3)(1 \: 2)$ \\
    $(2 \: 3) = (1 \: 2)(1 \: 3)(1 \: 2)$ \\
    $e = (1 \: 2)e(1 \: 2)$. \\

We have shown that every automorphism of $G$ can be written as an inner automorphism, and is thus an element of 
the inner automorphism group.  So $\mathrm{Inn}(G) \subseteq \mathrm{Aut}(G)$ and $\mathrm{Aut}(G) \subseteq \mathrm{Inn}(G)$, and as a result,  
$\mathrm{Inn}(G) = \mathrm{Aut}(G).$
\end{proof}

\section*{2}
Claim: The symmetric group $S_n$ is isomorphic to a subgroup of the alternating group $A_{n+2}$. 

\begin{proof}
    We define a map $f: S_n \rightarrow A_{n+2}$ as follows: for every permutation $\tau_j$ in $S_n$ for even $j$, $f(\tau_j) = \tau_j$, and for every permutation $\tau_k$ in $S_n$ for odd $k$, $f(\tau_k) = \tau_k \sigma$ where $\sigma$ is the transposition $ ( n \: \: \: n+1 )$.  Note that $\sigma$ and any $\tau$ are disjoint, and note that $\sigma$ is order 2.  We must show that this map is an isomorphism. \\

First, let two permutations $\tau_a$, $\tau_b$ in $S_n$ be given.  Their product $\tau_a \tau_b$ is either even or odd; if even, $f(\tau_a \tau_b) = \tau_a \tau_b$.  If the product is odd, $f(\tau_a \tau_b) = \tau_a \tau_b \sigma$, which produces an even permutation.  Now suppose $\tau_a$ is even and $\tau_b$ is even.  Then $f(\tau_a)f(\tau_b) = \tau_a \tau_b$.  If one of $\tau_a$, $\tau_b$ is even and the other odd, then the product of the mappings $f(\tau_a)f(\tau_b)$ is either $\tau_a \sigma \tau_b$ or $\tau_a \tau_b \sigma$, and since $\sigma$ is disjoint, these two terms are equal.  If both $\tau_a$ and $\tau_b$ are odd, then $f(\tau_a)f(\tau_b) = \tau_a \sigma \tau_b \sigma = \tau_a \sigma \sigma \tau_b = \tau_a \tau_b$.  So $f(\tau_a \tau_b) = f(\tau_a)f(\tau_b)$, and the mapping $f$ is a homomorphism. \\

Next, let two permutations $\tau_a$, $\tau_b$ be given, and suppose $f(\tau_a) = f(\tau_b)$.  We have two cases for $f(\tau_a)$.  Either $f(\tau_a) = \tau_a$, or $f(\tau_a) = \tau_a \sigma$.  In the former case, this implies $\tau_a = \tau_b$.  In the latter case, this implies $\tau_a \sigma = \tau_b \sigma$.  Then with $\sigma$ being order 2, we have 

\begin{alignat*}{1}
    \tau_a \sigma \sigma^{-1} &= \tau_b \sigma \sigma^{-1}  \\ 
    \rightarrow \tau_a \sigma \sigma &= \tau_b \sigma \sigma \\ 
    \rightarrow \tau_a &= \tau_b, \\
\end{alignat*}

so the mapping $f$ is injective. \\

To show that $f$ is surjective, let $\phi$ in $B \subseteq A_{n+2}$ be given.  Then $\phi$ is either of the form $\tau_a$ for even $a$, or $\tau_a \sigma$ for odd $a$.  Since $\tau_a$ is a permutation of at most length $n$, it is in the group $S_n$ and we can choose $\tau_a$ from $S_n$ to map $f(\tau_a) = \phi$, so the mapping $f$ is surjective. \\

By the properties of homomorphisms, $f$ maps the identity in $S_n$ to the identity in $B \subseteq A_{n+2}$, and likewise it maps all inverses to $B$.  Since $\sigma$ is of order 2, the product of any two elements $\phi$ and $\alpha$ in $B$ are of the form $\tau_a$ for even $a$, or $\tau_a \sigma$ for odd $a$, so $B$ is closed under the group operation and is a subgroup of $A_{n+2}$.
\end{proof}

\end{document}