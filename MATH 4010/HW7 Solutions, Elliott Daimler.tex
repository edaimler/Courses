%This is my super simple Real Analysis Homework template

\documentclass{article}
\usepackage[utf8]{inputenc}
\usepackage[english]{babel}
\usepackage[]{amsthm} %lets us use \begin{proof}
\usepackage[]{amssymb} %gives us the character \varnothing
\usepackage{fancyhdr}
\usepackage{amsmath}
\usepackage[margin=1in]{geometry} %reduce margins to 0.5 inch

\setlength{\parindent}{0pt}
\setlength{\parskip}{1em}

%\pagestyle{fancy}
\makeatletter
\renewcommand*\env@matrix[1][*\c@MaxMatrixCols c]{%
  \hskip -\arraycolsep
  \let\@ifnextchar\new@ifnextchar
  \array{#1}}
\makeatother

\renewcommand*\descriptionlabel[1]{\hspace\labelsep\normalfont #1}

\title{\textbf{Homework 8, Sylow Theory Solutions}}
\date{}
%This information doesn't actually show up on your document unless you use the maketitle command below

\fancypagestyle{firstpage}{%
  \rhead{
	Elliott Daimler \\
	MATH 4010 \\
	Dr. Chastkofsky \\ 
	MWF 10:20am \\
	\date\today
  }
  
}


\begin{document}
\maketitle %This command prints the title based on information entered above

\thispagestyle{firstpage}
%Section and subsection automatically number unless you put the asterisk next to them.
\section*{1}

Claim: 

\begin{proof}

\end{proof}


\section*{2}

Claim: Let $G$ be a finite abelian group, and let $p$ be a prime number.  If $p$ divides the order of $G$, then $G$ has 
an element of order $p$.

\begin{proof}
    By the Fundamental Theorem of Finite Abelian Groups, $G$ is isomorphic to a direct 
    sum of cyclic groups of prime power order, $G \cong P_1 \oplus P_2 \oplus \cdots \oplus P_k$.  
    Since each group $P_i$ in the direct sum is cyclic, it has an element $a$ of order $p_i$, where 
    $p_i$ is prime and $p_i = |P_i|$.  The order of the direct sum $P_1 \oplus P_2 \oplus \cdots \oplus P_k$ 
    is $|P_1| \cdot |P_2| \cdot \cdots \cdot |P_k| = p_1 p_2 \cdots p_k = |G|$, so any prime $p$ that 
    divides $G$ is equal to the order of some $P_i$ in $P_1 \oplus P_2 \oplus \cdots \oplus P_k$, and 
    consequently is equal to the order of the generator $a$ of the cyclic group $P_i$.  \\ 

    Assume without loss of generality that there is an element $b \in P_1 \oplus P_2 \oplus \cdots \oplus P_k$ 
    of the form $b = (0, 0, \cdots, a, \cdots, 0)$ where $a$ is in the $i^{th}$ tuple position and all other 
    positions have a value of zero, the identity for each cyclic group.  Then under the group operation of the 
    direct sum, the order of $b$ is also $p$.  
    Since $P_1 \oplus P_2 \oplus \cdots \oplus P_k$ is isomorphic to $G$, there must be an element of order 
    $p$ in $G$.  So if there is a prime $p$ that divides $G$, that prime is the order of an element in $P_1 \oplus P_2 \oplus \cdots \oplus P_k$ 
    and thus is the order of an element in $G$.
\end{proof}

\end{document}