%This is my super simple Real Analysis Homework template

\documentclass{article}
\usepackage[utf8]{inputenc}
\usepackage[english]{babel}
\usepackage[]{amsthm} %lets us use \begin{proof}
\usepackage[]{amssymb} %gives us the character \varnothing
\usepackage{fancyhdr}
\usepackage{amsmath}
\usepackage[margin=1in]{geometry} %reduce margins to 0.5 inch

\setlength{\parindent}{0pt}
\setlength{\parskip}{1em}

%\pagestyle{fancy}
\makeatletter
\renewcommand*\env@matrix[1][*\c@MaxMatrixCols c]{%
  \hskip -\arraycolsep
  \let\@ifnextchar\new@ifnextchar
  \array{#1}}
\makeatother

\renewcommand*\descriptionlabel[1]{\hspace\labelsep\normalfont #1}

\title{\textbf{Homework 10, Chapter 11 Solutions}}
\date{}
%This information doesn't actually show up on your document unless you use the maketitle command below

\fancypagestyle{firstpage}{%
  \rhead{
	Elliott Daimler \\
	MATH 4010 \\
	Dr. Chastkofsky \\ 
	MWF 10:20am \\
	\date\today
  }
  
}


\begin{document}
\maketitle %This command prints the title based on information entered above

\thispagestyle{firstpage}
%Section and subsection automatically number unless you put the asterisk next to them.
\section*{1}

The minimum polynomial of $\sqrt{2}$ over $\mathbb{Q}$ is $x^2 - 2$; since this polynomial 
is degree $2$,  $[\mathbb{Q}(\sqrt{2}):\mathbb{Q}] = 2$.  
Similarly, the minimum polynomial of 
$\sqrt{5}$ over $\mathbb{Q}$ is $x^2 - 5$.  The roots here are not in $\mathbb{Q}(\sqrt{2})$, 
so this polynomial is irreducible over $\mathbb{Q}(\sqrt{2})$ and 
$[\mathbb{Q}(\sqrt{5}):\mathbb{Q}] = 2$.  The minimum polynomial of $\sqrt{10}$ 
over $\mathbb{Q}$ is $x^2 - 10$; this polynomial has roots in $\mathbb{Q}(\sqrt{2})(\sqrt{5})$, 
namely $(\pm \sqrt{2} \sqrt{5})$.  The result is  
$[\mathbb{Q}(\sqrt{2}, \sqrt{5}, \sqrt{10}):\mathbb{Q}] = 4$. 


\section*{2} 
The polynomial $x^4 - 4x^2 - 5$ can be factored into $(x^2 - 5)(x^2 + 1)$; 
each of these polynomials can be further factored into $x^2 - 5 = (x + \sqrt{5})(x - \sqrt{5})$ 
and $x^2 + 1 = (x + i)(x - i)$.  Since the polynomial is degree $4$ and 
we've factored into four distinct polynomials of degree one, we have all of the roots: 
$\pm i$, $\pm \sqrt{5}$.  Since negative coefficients are in the field extensions
$\mathbb{Q}(\sqrt{5})$ and $\mathbb{Q}(i)$, the splitting field is $\mathbb{Q}(\sqrt{5}, i)$.

\section*{3}
\subsection*{a}
Claim: If $f(x) = cx^n \in F[x]$ and $g(x) = b_0 + b_1x^1 + \cdots + b_k x^k \in F[x]$, 
then $(fg)' (x) = f(x)g'(x) + f'(x)g(x)$.

\begin{proof}
    We have $(fg)(x) = \sum _{i=0} ^k cb_ix^{n+i}$, so 
    $(fg)'(x) = (n+i)\sum_{i=1} ^k cb_i x^{n+i-1}$.  \\ 
    \\ 
    Taking the individual functions' derivatives, we have 
    $f'(x) = cnx^{n-1}$ and $g'(x) = \sum_{i=1} ^k b_ix^{i-1}$.  So 
    $fg'(x) = \sum cb_i x^{n+i-1}(i)$ and $f'g(x) = \sum_{i=1}^k cnb_ix^{n+i-1}$, and 
    $fg'(x) + f'g(x) = (n+i)\sum_{i=1} ^k cb_i x^{n+i-1} = (fg)'(x)$.
\end{proof}

\subsection*{b}

Claim: If $f(x)$, $g(x)$ are any polynomials in $F[x]$, then $(fg)'(x) = f(x)g'(x) + f'(x)g(x)$.

\begin{proof}
    Let $f(x)$ be of the form $f(x) = a_0 + a_1 x^1 + \cdots + a_n x^n$.  Then this is the 
    same form of $g(x)$ from part \textit{a}.  So $(fg)(x)$ can be written as 
    $(fg)(x) = \sum _{i=0} ^n a_i x^i g(x)$.  We then have for each term 
    $(a_i x^i g(x))' = i a_i x^{i-1}g(x) + a_i x^i g'(x)$, so by the summation rule, 
    $(fg)'(x) = (\sum _{i=0} ^n a_i x^i g(x))' = \sum_{i=1}^n i a_i x^{i-1}g(x) + 
    \sum_{i=0}^n a_i x^i g'(x) = fg'(x) + f'g(x)$.
\end{proof}

\end{document}