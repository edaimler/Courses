%This is my super simple Real Analysis Homework template

\documentclass{article}
\usepackage[utf8]{inputenc}
\usepackage[english]{babel}
\usepackage[]{amsthm} %lets us use \begin{proof}
\usepackage[]{amssymb} %gives us the character \varnothing
\usepackage{fancyhdr}
\usepackage{amsmath}
\usepackage[margin=1in]{geometry} %reduce margins to 0.5 inch

\setlength{\parindent}{0pt}
\setlength{\parskip}{1em}

%\pagestyle{fancy}
\makeatletter
\renewcommand*\env@matrix[1][*\c@MaxMatrixCols c]{%
  \hskip -\arraycolsep
  \let\@ifnextchar\new@ifnextchar
  \array{#1}}
\makeatother

\renewcommand*\descriptionlabel[1]{\hspace\labelsep\normalfont #1}

\title{\textbf{Homework 8, Sylow Theory Solutions}}
\date{}
%This information doesn't actually show up on your document unless you use the maketitle command below

\fancypagestyle{firstpage}{%
  \rhead{
	Elliott Daimler \\
	MATH 4010 \\
	Dr. Chastkofsky \\ 
	MWF 10:20am \\
	\date\today
  }
  
}


\begin{document}
\maketitle %This command prints the title based on information entered above

\thispagestyle{firstpage}
%Section and subsection automatically number unless you put the asterisk next to them.
\section*{1}

Let $G = S_9$.  Elements in the centralizer send conjugates of $x$ to $x$; in other words, 
for elements $g$ in the centralizer of $x$, $gxg^{-1} = x$.  Permutation cycles preserve their structure under conjugation, so 
the centralizer of $h = (1 \quad 2 \quad 3)(4 \quad 5 \quad 6)(7 \quad 8 \quad 9)$ will preserve its structure.
Then the centralizer of $(1 \quad 2 \quad 3)(4 \quad 5 \quad 6)(7 \quad 8 \quad 9)$ is the set of permutations in $G$ that 
permute the elements within each disjoint cycle, i.e. \\
$g (1 \quad 2 \quad 3)(4 \quad 5 \quad 6)(7 \quad 8 \quad 9) g^{-1} = 
g (1 \quad 2 \quad 3)g^{-1}g(4 \quad 5 \quad 6)g^{-1}g(7 \quad 8 \quad 9) g^{-1}$.
The cycles in the conjugation must therefore be of the same disjoint form with elements 
$1, 2, 3$ in a cycle, $4, 5, 6$ in a cycle, and $7, 8, 9$ in a cycle.  For example, 
$(1 \quad 2 \quad 3)(1 \quad 2 \quad 3)(4 \quad 5 \quad 6)(7 \quad 8 \quad 9)(1 \quad 3 \quad 2) = 
(1 \quad 2 \quad 3)(4 \quad 5 \quad 6)(7 \quad 8 \quad 9)$, so $(1 \quad 2 \quad 3)$ is 
in the centralizer.  We find that the three disjoint cycles generate the centralizer, 
or $C_{S_9}(h) = <(1 \quad 2 \quad 3), (4 \quad 5 \quad 6), (7 \quad 8 \quad 9)>$.  Each of these 
dijoint cycles has order 3, and there are $3(3) = 9$ combinations of these elements, 
so the centralizer contains a Sylow 3-subgroup of $G$.


\section*{2}

The three-cycles generating the centralizer in part 1 are even permutations, so are in 
$A_9$.  Again, they have order three and there are 9 combinations of the generators, 
so the centralizer is a Sylow 3-subgroup of $G$.

\section*{3}

We will use the approach used in a lemma on interactions of p-Sylows.  
Let $Q$ be a Sylow-p subgroup of a group $G$, and let $P$ be a p-subgroup of $G$.  
For the normalizer $N$ of $Q$, let $H = P \cap N$.  $Q$ is a normal subgroup of $N$, 
so $HQ$ is a subgroup on $N$.  Then $HQ/Q$ is isomorphic to $H/(H \cap Q)$ by the 
Second Isomorphism Theorem.  $Q$ is a Sylow-p subgroup of $G$, so its order 
is the highest power of $p$ that divides the order of $G$; as a result, $p$ does 
not divide the index $[HQ:Q]$.  $H$ is a subgroup of $P$, so $p$ is the only prime 
dividing the index $[H:H \cap Q]$.  Since these two quotient groups are isomorphic, 
and $p$ divides one and not the other, their order must be 1.  It follows that $HQ = Q$, 
so any elements of $P$ that are in the normalizer of $Q$ are also in $Q$.

\newpage

\section*{4}

For Sylow-p subgroup $Q$ of $G$ and p-subgroup $P$ of $G$, let $H = \{gQg^{-1} | g \in G\}$, 
or the set of conjugates of $Q$ in $G$.  For $P \cap H = \{pQp^{-1} | p \in P \}$, 
the orbits have orders of $p^n$ for integer $n$.  As with the proof of the Second 
Sylow Theorem given in our notes, there is a least one orbit of length 1, so there is 
some $g \in G$ where for every $x \in P$, $x(gQg^{-1})x^{-1} = gQg^{-1}$.  
The conjugate of $P$ is in the normalizer of $Q$, and so it is also in $Q$.






\end{document}