%This is my super simple Real Analysis Homework template

\documentclass{article}
\usepackage[utf8]{inputenc}
\usepackage[english]{babel}
\usepackage[]{amsthm} %lets us use \begin{proof}
\usepackage[]{amssymb} %gives us the character \varnothing
\usepackage{fancyhdr}
\usepackage{amsmath}
\usepackage[margin=1in]{geometry} %reduce margins to 0.5 inch

\setlength{\parindent}{0pt}
\setlength{\parskip}{1em}

%\pagestyle{fancy}
\makeatletter
\renewcommand*\env@matrix[1][*\c@MaxMatrixCols c]{%
  \hskip -\arraycolsep
  \let\@ifnextchar\new@ifnextchar
  \array{#1}}
\makeatother

\renewcommand*\descriptionlabel[1]{\hspace\labelsep\normalfont #1}

\title{\textbf{Homework 3, Section 7.4 Solutions}}
\date{}
%This information doesn't actually show up on your document unless you use the maketitle command below

\fancypagestyle{firstpage}{%
  \rhead{
	Elliott Daimler \\
	MATH 4010 \\
	Dr. Chastkofsky \\ 
	MWF 10:20am \\
	\date\today
  }
  
}


\begin{document}
\maketitle %This command prints the title based on information entered above

\thispagestyle{firstpage}
%Section and subsection automatically number unless you put the asterisk next to them.
\section*{1}

Claim: The automorphism group of the additive group $\mathbb{Z}_n$ is isomorphic to $\mathbb{Z}_n^*$.

\begin{proof}
    We will show that an isomorphism exists from $\mathbb{Z}_n^*$ to $\mathrm{Aut}(\mathbb{Z}_n)$.  Let 
    $f:\mathbb{Z}_n^* \rightarrow \mathrm{Aut}(\mathbb{Z}_n)$ be the map $k \rightarrow \varphi_k$ for $k \in \mathbb{Z}_n^*$, defined such that  
    for $[a] \in \mathbb{Z}_n$, $\varphi_k([a]) = [ka]$.\\
    
    1) We must first show that $\varphi_k$ is 
    a well-defined bijective homomorphism from $\mathbb{Z}_n$ to $\mathbb{Z}_n$.  Let $[a] = [b]$ for 
    $[a], [b] \in \mathbb{Z}_n$.  We have $\varphi_k([a]) = [ka]$ and $\varphi_k([b]) = [kb]$.  
    In the ring $\mathbb{Z}_n$, $[ka] = [k][a]$ and $[kb] = [k][b]$.  Since 
    $[a] = [b]$, it follows that $[k][a] = [k][b]$, so $\varphi_k$ is well-defined.  \\ 

    Next, let $\varphi_k([a]) = \varphi_k([b])$.  Then $\varphi_k([a]) = [ka] = [kb] = \varphi_k([b])$.  
    By the cancellation property, $[ka] = [k][a] = [k][b] = [kb]$ implies $[a] = [b]$, so 
    $\varphi_k$ is injective.\\ 

    To show that $\varphi_k$ is surjective, let $[c] \in \mathbb{Z}_n$ and $[k] \in 
    \mathbb{Z}_n^*$ be given.  We must show that there is some $[a] \in \mathbb{Z}_n$ such that 
    $\varphi_k([a]) = [c]$.  Since $k$ is relatively prime to $n$, there exist some $u, v \in \mathbb{Z}_n$ 
    where $ku + vn = 1$.  Multiplying both sides of this equation on the right by $c$ yields $(ku + vn)c = kuc + vnc = c$.  
    Now we can choose some $a \in \mathbb{Z}_n$ where $a = uc$.  Since $vnc = 0$ in $\mathbb{Z}_n$, 
    we have  
    \begin{alignat*}{1}
        kuc + vnc &= c \\
    \rightarrow ka + 0 &= c \\
    \rightarrow ka &= c, \tag{1}\label{c} 
    \end{alignat*} 
        
        
    so the map $\varphi_k$ is surjective.

    Now let $a, b \in \mathbb{Z}_n$ be given.  Then $\varphi_k([a + b]) = [k(a+b)] = [k][a+b]$.  Similarly, 
    $\varphi_k([a]) + \varphi_k([b]) = [ka] + [kb] = [k]([a] + [b]) = [k][a+b]$.  Since the group operation of $\mathbb{Z}_n$ 
    is addition, and $\varphi_k([a+b]) = \varphi_k([a]) + \varphi_k([b])$, the map $\varphi_k$ is a homomorphism to $\mathbb{Z}_n$.\\ 

    2) Next, we must show that the map $f$ defined as $k \rightarrow \varphi_k$ is a well-defined bijective homomorphism from 
    $\mathbb{Z}_n^*$ to $\mathrm{Aut}(\mathbb{Z}_n)$.  To show that $f$ is well-defined, let $k = j \in \mathbb{Z}_n^*$ 
    be given.  We have $\varphi([a]) = [ka] = [k][a]$ and $\varphi([a]) = [ja] = [j][a]$.  Since $[k] = [j]$, 
    it follows that $[k][a] = [ka] = [ja] = [j][a]$, so the map $f$ is well-defined. \\ 

    To show that $f$ is injective, let $\varphi_j = \varphi_k \in \mathrm{Aut}(\mathbb{Z}_n)$ be given.  Then for some $a \in \mathbb{Z}_n$, 
    we have $\varphi_j([a]) = \varphi_k([a])$ and so $[ja] = [j][a] = [k][a] = [ka]$.  By the cancellation property, 
    $[j][a] = [k][a]$ implies $[j] = [k]$, so the map $f$ is injective. \\ 

    We will now show that $f$ is surjective.  Let $\varphi_c \in \mathrm{Aut}(\mathbb{Z}_n)$ be 
    given.  Then for some $b \in \mathbb{Z}_n$, $\varphi_c([b]) = [cb]$.  There is some 
    $[a] \in \mathbb{Z}_n$ where $[a] = [cb]$.  Using a similar argument used for 
    equation \eqref{c}, we can be given some $k \in \mathbb{Z}_n^*$ and choose some $[d] \in \mathbb{Z}_n$ 
    where $[kd] = [a] = [cb]$. So for any $\varphi_c \in \mathrm{Aut}(\mathbb{Z}_n)$, there is some $k \in \mathbb{Z}_n^*$ 
    such that $\varphi_k = \varphi_c$, so $f$ is surjective. \\ 

    Finally, let $k, j \in \mathbb{Z}_n^*$ be given.  Then for some $a \in \mathbb{Z}_n$, 
    $\varphi_{kj}([a]) = [kja]$.  Similarly, the composition $\varphi_k \circ \varphi_j ([a])$ 
    yields $\varphi_k([ja]) = [kja]$.  The group operation of $\mathrm{Aut}(\mathbb{Z}_n)$ is composition, 
    so the map $f$ is a homomorphism to $\mathrm{Aut}(\mathbb{Z}_n)$. \\ 

    we have shown that for each $k \in \mathbb{Z}_n^*$, $\varphi_k$ is isomorphic from $\mathbb{Z}_n$ to 
    $\mathbb{Z}_n$, and we have shown that the $f$, the map $k \rightarrow \varphi_k$ is isomorphic 
    from $\mathbb{Z}_n^*$ to $\mathrm{Aut}(\mathbb{Z}_n)$, so we have constructed an isomorphism from 
    $\mathbb{Z}_n$ to $\mathrm{Aut}(\mathbb{Z}_n)$ and we are done.
\end{proof}


\end{document}