%This is my super simple Real Analysis Homework template

\documentclass{article}
\usepackage[utf8]{inputenc}
\usepackage[english]{babel}
\usepackage[]{amsthm} %lets us use \begin{proof}
\usepackage[]{amssymb} %gives us the character \varnothing
\usepackage{fancyhdr}
\usepackage{amsmath}
\usepackage[margin=1in]{geometry} %reduce margins to 0.5 inch

\setlength{\parindent}{0pt}
\setlength{\parskip}{1em}

%\pagestyle{fancy}
\makeatletter
\renewcommand*\env@matrix[1][*\c@MaxMatrixCols c]{%
  \hskip -\arraycolsep
  \let\@ifnextchar\new@ifnextchar
  \array{#1}}
\makeatother

\renewcommand*\descriptionlabel[1]{\hspace\labelsep\normalfont #1}

\title{\textbf{Bonus Exercise on Linear Groups}}
\date{}
%This information doesn't actually show up on your document unless you use the maketitle command below

\fancypagestyle{firstpage}{%
  \rhead{
	Elliott Daimler \\
	MATH 4010 \\
	Dr. Chastkofsky \\ 
	MWF 10:20am \\
	\date\today
  }
  
}


\begin{document}
\maketitle %This command prints the title based on information entered above

\thispagestyle{firstpage}
%Section and subsection automatically number unless you put the asterisk next to them.
\section*{1}

Claim: The automorphism group of $\mathbb{Z}_2 \oplus \mathbb{Z}_2 \oplus \mathbb{Z}_2$ 
is isomorphic to $GL(3, 2)$.

\begin{proof}
    For readability, call $G = \mathbb{Z}_2 \oplus \mathbb{Z}_2 \oplus \mathbb{Z}_2$.  The group 
    $G$ is generated by $(1, 0, 0), (0, 1, 0), (0, 0, 1)$, which, when treated as vectors in $\mathbb{Z}_2^3$, 
    form a basis.  The set of non-singular matrices in $\mathrm{GL}(3,2)$ are closed under multiplication, and due to their non-singularity are closed under inverses.  
Their kernel is the zero vector $(0,0,0)$, which is the identity element in $G$.  The composition of non-singular matrices in $\mathrm{GL}(3,2)$ 
with a vector with entries in $\mathbb{Z}_2$ span the vector space $\mathbb{Z}_2^3$, so $\mathrm{GL}(3,2)$ is isomorphic to $\mathrm{ Aut }(G)$.
\end{proof}


\end{document}




