%This is my super simple Real Analysis Homework template

\documentclass{article}
\usepackage[utf8]{inputenc}
\usepackage[english]{babel}
\usepackage[]{amsthm} %lets us use \begin{proof}
\usepackage[]{amssymb} %gives us the character \varnothing
\usepackage{fancyhdr}
\usepackage{amsmath}
\usepackage[margin=1in]{geometry} %reduce margins to 0.5 inch

\setlength{\parindent}{0pt}
\setlength{\parskip}{1em}

%\pagestyle{fancy}
\makeatletter
\renewcommand*\env@matrix[1][*\c@MaxMatrixCols c]{%
  \hskip -\arraycolsep
  \let\@ifnextchar\new@ifnextchar
  \array{#1}}
\makeatother

\renewcommand*\descriptionlabel[1]{\hspace\labelsep\normalfont #1}

\title{\textbf{Homework 11, Section 12.1 Solutions}}
\date{}
%This information doesn't actually show up on your document unless you use the maketitle command below

\fancypagestyle{firstpage}{%
  \rhead{
	Elliott Daimler \\
	MATH 4010 \\
	Dr. Chastkofsky \\ 
	MWF 10:20am \\
	\date\today
  }
  
}


\begin{document}
\maketitle %This command prints the title based on information entered above

\thispagestyle{firstpage}
%Section and subsection automatically number unless you put the asterisk next to them.
\section*{1}

Claim: If $K$ is an extension field of $\mathbb{Q}$ and $\sigma$ is an automorphism of $K$, then 
$\sigma$ is a $\mathbb{Q}$-automorphism.

\begin{proof}
    $K$ is characteristic $0$, so $1 \cdot n = \underbrace{1 + 1 + \cdots + 1}_\text{n}$ and 
    $\sigma(n) = \underbrace{\sigma(1) + \sigma(1) + \cdots + \sigma(1)}_\text{n} = n$.  
    Elements of the field $\mathbb{Q}$ have the form $n/m = n(m^{-1})$ where $n, m \in \mathbb{Z}$ and  
    $m^{-1}$ is the multiplicative inverse of $m$.  \\ 

    Now let $a \in \mathbb{Q}$ be given.  Then there are some $n, m \in \mathbb{Z}$ where 
    $a = n(m^{-1})$ and 
    \begin{alignat*}{1}
      \sigma(a) &= \sigma[n(m^{-1})] \\ 
      &= \sigma(n)\sigma(m^{-1}) \\ 
      &= \sigma(n)\sigma(m)^{-1}. \\ 
    \end{alignat*}

    Making further use of the properties of automorphism, we have 

    \begin{alignat*}{1}
      \sigma(n)\sigma(m)^{-1} &= \underbrace{\sigma(1) + \sigma(1) + \cdots + \sigma(1)}_\text{n} \underbrace{(\sigma(1) + \sigma(1) + \cdots + \sigma(1))^{-1}}_\text{m} \\ 
      &= n(m^{-1}).
    \end{alignat*}

    So $\sigma$ maps the rational $a$ to itself, and is a $\mathbb{Q}$-automorphism.

\end{proof}


\section*{2} 

\subsection*{a}

For $\omega = (-1+ \sqrt{3} i)/2$, we have 
\begin{alignat*}{1}
    \omega^2 &= (-1 + \sqrt{3}i)(-1+ \sqrt{3}i)/4 \\ 
    &= (1 - 2 \sqrt{3} i - 3)/4 \\ 
    &= (-2 - 2 \sqrt{3}i)/4 \\ 
    &= (-1 - \sqrt{3}i)/2, \\ 
\end{alignat*}

so $\omega^2$ is the complex conjugate of $\omega$.  Using a similar computation 
to the above, we have $(\omega^2)^2 = \omega$.
The minimum polynomial of $\omega$ 
over $\mathbb{Q}$ is $x^2 + x + 1$: \\ 

\begin{alignat*}{1}
    \omega^2 + \omega + 1 &= (-1 - \sqrt{3}i)/2 + (-1 + \sqrt{3}i)/2 + 1 \\ 
    &= \frac{-1}{2} - \frac{\sqrt{3}i}{2} + \frac{-1}{2} + \frac{\sqrt{3}i}{2} + 1 \\ 
    &= -1 + 1 \\ 
    &= 0. \\ 
\end{alignat*}

Since $(\omega^2)^2 = \omega$, if we use $\omega^2$ in the above polynomial, 
the two complex conjugates simply switch positions and we get the same result:

\begin{alignat*}{1}
  (\omega ^2)^2 + (\omega ^2) + 1 &= (-1 + \sqrt{3}i)/2 + (-1 - \sqrt{3}i)/2 + 1 \\ 
  &= \frac{-1}{2} + \frac{\sqrt{3}i}{2} + \frac{-1}{2} - \frac{\sqrt{3}i}{2} + 1 \\ 
  &= 0. \\ 
\end{alignat*}

So $\omega$ and $\omega^2$ are the roots of this polynomial.

\newpage

\subsection*{b}

The minimal polynomial in part \textbf{a} has degree 2, so by the 
Fundamental Theorem of Algebra, there are two roots.  We have already identified 
these roots as $\omega = (-1+ \sqrt{3} i)/2$ and $\omega^2 = (-1 - \sqrt{3} i)/2$, 
so the Galois group $\mathrm{Gal}_{\mathbb{Q}}\mathbb{Q}(\omega)$ has one 
automorphism, say $\sigma$, that permutes these two roots,
 and one identity automorphism, say $\tau$: 

\begin{alignat*}{1}
  \sigma(\omega) = \omega^2 \\ 
  \sigma(\omega^2) = \omega \\ 
  \tau(\omega) = \omega \\ 
  \tau(\omega^2) = \omega^2. \\
\end{alignat*}

The above mappings obey the properties of automorphism; for example, 
with $\sigma(\omega^2)$ we have  
$\sigma(\omega^2) = \sigma(\omega)^2 = \sigma(\omega)\sigma(\omega) = \omega^2 \omega^2$, 
and as shown in part \textbf{a}, $(\omega^2)^2 = \omega$.  So $\mathrm{Gal}_{\mathbb{Q}}\mathbb{Q}(\omega)$ 
has the two elements described above.

\end{document}