%This is my super simple Real Analysis Homework template

\documentclass{article}
\usepackage[utf8]{inputenc}
\usepackage[english]{babel}
\usepackage[]{amsthm} %lets us use \begin{proof}
\usepackage[]{amssymb} %gives us the character \varnothing
\usepackage{fancyhdr}
\usepackage{amsmath}
\usepackage[margin=1in]{geometry} %reduce margins to 0.5 inch

\setlength{\parindent}{0pt}
\setlength{\parskip}{1em}

%\pagestyle{fancy}
\makeatletter
\renewcommand*\env@matrix[1][*\c@MaxMatrixCols c]{%
  \hskip -\arraycolsep
  \let\@ifnextchar\new@ifnextchar
  \array{#1}}
\makeatother

\renewcommand*\descriptionlabel[1]{\hspace\labelsep\normalfont #1}

\title{\textbf{Homework 2, Section 7.3 Solutions}}
\date{}
%This information doesn't actually show up on your document unless you use the maketitle command below

\fancypagestyle{firstpage}{%
  \rhead{
	Elliott Daimler \\
	MATH 4010 \\
	Dr. Chastkofsky \\ 
	MWF 10:20am \\
	\date\today
  }
  
}


\begin{document}
\maketitle %This command prints the title based on information entered above

\thispagestyle{firstpage}
%Section and subsection automatically number unless you put the asterisk next to them.
\section*{1}

Claim: If $a$ is the only element of a group $G$ whose order is 2, then $a$ is in the center $G$.

\begin{proof}
    Let $b \in G$ be given.  By the closure property, there is some $c \in G$ such that 
    $c = b^{-1}ab$.  Then for $c^2$ we have
    \begin{alignat*}{1}
        c^2 &= (b^{-1}ab)^2 \\ 
        &= b^{-1}abb^{-1}ab \\ 
        &= b^{-1}aeab \\ 
        &= b^{-1}aab \\ 
        &= b^{-1}eb \\ 
        &= b^{-1}b \\ 
        &= e. \\
    \end{alignat*} 
    The order of $c$ is 2, and only $a \in G$ has order 2, so $c = a$ and $a = b^{-1}ab$.  
    Applying the group operation with $b$ on the left of both terms of this equation gives 
    \begin{alignat*}{1}
        ba &= bb^{-1}ab \\ 
        &= eab \\ 
        &= ab, \\ 
    \end{alignat*}
    so the equation $c = b^{-1}ab$ implies $ba = ab$, and $a$ is in the center of $G$.
\end{proof}

\section*{2}

Claim: Let $p$ be a prime integer.  Then for any integer $a$, $a^p \equiv a \mod{p}$.

\begin{proof} 
    First we will show that for any $b$ in $\mathbb{Z}^*_p$, $b^{p-1} = 1$.  Since $\mathbb{Z}^*_p$ 
    is a finite multiplicative group, by Theorem 7.16 $\mathbb{Z}^*_p$ is cyclic.  So there is some $c \in \mathbb{Z}^*_p$ 
    that generates $\mathbb{Z}^*_p$, or $<c> = \mathbb{Z}^*_p$.  Since $p$ is prime, every nonzero element of $\mathbb{Z}_p$ 
    is relatively prime to $p$, so $\mathbb{Z}^*_p$ has $p-1$ elements and $|c| = p-1$ is the highest order 
    of an element in $\mathbb{Z}^*_p$.

    Being a cyclic group, $\mathbb{Z}^*_p$ is abelian, so by Corollary 7.10, $|b|$ divides $p-1$.  If follows that 
    $b^{p-1} = 1$.

    Now let $b$ be the congruence class of $a$ in $\mathbb{Z}^*_p$.  Then $a^{p-1} \equiv 1 \mod{p}$ and 
    $aa^{p-1} = a^p \equiv a \mod{p}$.
\end{proof}

\end{document}