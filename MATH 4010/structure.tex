%This is my super simple Real Analysis Homework template

\documentclass{article}
\usepackage[utf8]{inputenc}
\usepackage[english]{babel}
\usepackage[]{amsthm} %lets us use \begin{proof}
\usepackage[]{amssymb} %gives us the character \varnothing
\usepackage{fancyhdr}

\pagestyle{fancy}

\title{Homework 1}
\author{Your Name}
\date\today
%This information doesn't actually show up on your document unless you use the maketitle command below

\fancypagestyle{firstpage}{%
  \lhead[L]{**Left Header for just the first page**}
  check
  \rhead[R]{**Right Header for just the first page**}
}


\begin{document}
\maketitle %This command prints the title based on information entered above

\thispagestyle{firstpage}
%Section and subsection automatically number unless you put the asterisk next to them.
\section*{Section 2.1}
Let $m:\mathcal{A}\rightarrow [0,\infty)$ be a set function where $\mathcal{A}$ is a $\sigma$-algebra. Assume $m$ is countably additive over countable disjoint collections of sets in $\mathcal{A}$.
%Basically, you type whatever text you want and use the $ sign to enter "math mode".
%For fancy calligraphy letters, use \mathcal{}
%Special characters are their own commands

\subsection*{Problem 1}
Given sets $A$, $B$, and $C$, if $A\subset B \textrm{ and } B \subset C$, then $A \subset C$.
\begin{proof}
Other symbols you can use for set notation are
\begin{itemize}
\item$A \supset B \supseteq C \subset D \subseteq E$. Also $\varnothing \textrm{vs} \emptyset$
\item$\cup$ and $\cup_{k=1}^\infty E_k$
\item$\cap$ and $\cap_{x \in \mathbb{N}} \{\frac{1}{\sqrt[3]{x}}\}$
\item$\bigcup$ and $\bigcap\limits_{k=0}^n$ and $\bigcap$
\item most Greek letters $\sigma \pi \theta \lambda_i e^{i\pi}$
\item $\int_0^2 ln(2)x^2sin(x) dx$
\item$\leq < \geq > = \neq$
\end{itemize}
If you want centered math on its own line, you can use a slash and square bracket.\\
\[
\left \{
\sum\limits_{k=1}^\infty l(I_k):A\subseteq \bigcup_{k=1}^\infty \{I_k\}
\right \}
\]
The left and right commands make the brackets get as big as we need them to be.
\end{proof}

\clearpage %Gives us a page break before the next section. Optional.
\subsection*{Problem 2}
Given...
\begin{proof}
Let $\epsilon>0$.
If you have a shorter statement that you still want centered, use two \$\$ on either side.
$$\exists \textrm{ some } \delta>0 \mid ...$$
\end{proof}

\subsection*{Problem 3}
%
\begin{proof}
%
\end{proof}

\section*{Section 2.2}
%
\subsection*{Problem 6}
Blah
\subsection*{Problem 7}
Blah
\subsection*{Problem 10}
Blah

\end{document}