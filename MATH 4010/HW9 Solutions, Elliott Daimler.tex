%This is my super simple Real Analysis Homework template

\documentclass{article}
\usepackage[utf8]{inputenc}
\usepackage[english]{babel}
\usepackage[]{amsthm} %lets us use \begin{proof}
\usepackage[]{amssymb} %gives us the character \varnothing
\usepackage{fancyhdr}
\usepackage{amsmath}
\usepackage[margin=1in]{geometry} %reduce margins to 0.5 inch

\setlength{\parindent}{0pt}
\setlength{\parskip}{1em}

%\pagestyle{fancy}
\makeatletter
\renewcommand*\env@matrix[1][*\c@MaxMatrixCols c]{%
  \hskip -\arraycolsep
  \let\@ifnextchar\new@ifnextchar
  \array{#1}}
\makeatother

\renewcommand*\descriptionlabel[1]{\hspace\labelsep\normalfont #1}

\title{\textbf{Homework 9, Section 9.4 Solutions}}
\date{}
%This information doesn't actually show up on your document unless you use the maketitle command below

\fancypagestyle{firstpage}{%
  \rhead{
	Elliott Daimler \\
	MATH 4010 \\
	Dr. Chastkofsky \\ 
	MWF 10:20am \\
	\date\today
  }
  
}


\begin{document}
\maketitle %This command prints the title based on information entered above

\thispagestyle{firstpage}
%Section and subsection automatically number unless you put the asterisk next to them.
\section*{1}

Let $N$ be a normal subgroup of $G$, let $a \in G$, and let $C$ be the conjugacy 
class of $a$ in $G$. 

\subsection*{a)} 
Claim: $a \in N$ if and only if $C \subseteq N$. 

\begin{proof}
    Suppose $a \in N$.  Since $N$ is normal,
     for any $h = g^{-1}ag$ for $g \in G$, $h \in N$, so all conjugates of 
     $a$ are in $N$. \\ 
     Conversely, suppose $C \subseteq N$.  Then there is some $h \in N$ where 
     $h = g^{-1}ag$.  Again, since $N$ is normal, for any $x \in G$, $xNx^{-1} = N$, 
     so $xhx^{-1} = xx^{-1}axx^{-1} = a \in N$.
\end{proof}

\subsection*{b)} 
Claim: If $C_i$ is any conjugacy class in $G$, prove that $C_i \subseteq N$ 
or $C_i \cap N = \emptyset$.

\begin{proof}
    Assume there is some non-identity $h \in N$ and $h \in C_i$.  Now assume 
    for the purpose of contradiction that $C_i \nsubseteq N$.  Then there is 
    some $x \in G$ such that for any $y \in C_i$, $g^{-1}hg = y$ for $g \in G$.  
    Since $g^{-1}Ng = N$, $g^{-1}hg \in N$ which implies that $y \in N$.  So all 
    elements in $C_i$ are in $N$, and we have a contradiction.
\end{proof}

\subsection*{c)} 
The class equation shows that for a group $G$, $|G| = |C_1| + |C_2| + \cdots + |C_i|$.  
We've shown that each class equation belongs to one and only one normal subgroup of $G$, 
so it follows that $|N| = |C_1| + |C_2| + \cdots + |C_i|$ where $C_1, C_2, \cdots, C_i$ 
are the conjugacy classes contained in $N$.

\section*{2}

Claim: If $N \neq <e>$ is a normal subgroup of $G$ and $|G| = p^n$, $N \cap Z(G) \neq <e>$.

\begin{proof}
    By the First Sylow Theorem, $G$ is a Sylow p-group, so its subgroups are p-subgroups and 
    thus have non-trivial centers.  Consider a normal subgroup $M$ of one of these p-subgroups $P$;
    the conjugation of $M$ by $g \in G$ then maps $a \in Z(P)$ to itself, and $a$ is in $M$ 
    so the intersection is non-trivial.
\end{proof}

\section*{3}

Claim: If $K$ is a Sylow p-subgroup of $G$ and $H$ is a subgroup that contains the 
normalizer $N(K)$, then $[G:H] \equiv 1 \mod p$.

\begin{proof}
    By Theorem 9.25, the number of distinct H-conjugates of $K$ is $[H:H \cap N(K)]$.  
    Since $H$ contains $N(K)$, $H \cap N(K) = N(K)$.  The index $[G:N(K)]$ is equal to 
    $1 \mod p$, and $[H:N(K)] = 1 \mod p$, so $[G:N(K)] = [G:H][H:N(K)] = [G:H] \cdot 1 \mod p =
    1 \mod p$.  

\end{proof}


\section*{4}

Claim: If $K$ is a Sylow p-subgroup of a group $G$, then its $N(N(K)) = N(K)$ where 
$N()$ is a normalizer.

\begin{proof}
    Since $K$ is a Sylow p-subgroup of both $N(K)$ and $N(N(K))$, then for some 
    $g \in N(N(K))$, $g^{-1}Kg$ is a subgroup of $g^{-1}N(K)g$.  $N(K)$ is normal, so 
    $g^{-1}N(K)g = N(K)$.  By the conjugation property of Sylow p-groups, there is some 
    $h \in N(K)$ where $g^{-1}Kg = h^{-1}Kh$.  Since $h$ is in the normalizer of $K$, its 
    conjugation of $K$ remains in $K$, or $g^{-1}Kg = h^{-1}Kh = K$.  It follows that 
    $g$ is also in $N(K)$.
\end{proof}






\end{document}