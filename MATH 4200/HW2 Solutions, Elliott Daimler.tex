%This is my super simple Real Analysis Homework template

\documentclass{article}
\usepackage[utf8]{inputenc}
\usepackage[english]{babel}
\usepackage[]{amsthm} %lets us use \begin{proof}
\usepackage[]{amssymb} %gives us the character \varnothing
\usepackage{fancyhdr}
\usepackage{amsmath}
\usepackage[margin=1in]{geometry} %reduce margins to 0.5 inch

\setlength{\parindent}{0pt}
\setlength{\parskip}{1em}

%\pagestyle{fancy}
\makeatletter
\renewcommand*\env@matrix[1][*\c@MaxMatrixCols c]{%
  \hskip -\arraycolsep
  \let\@ifnextchar\new@ifnextchar
  \array{#1}}
\makeatother

\newcommand{\Tau}{\mathrm{T}}

\renewcommand*\descriptionlabel[1]{\hspace\labelsep\normalfont #1}

\title{\textbf{Section 2.2 Solutions}}
\date{}
%This information doesn't actually show up on your document unless you use the maketitle command below

\fancypagestyle{firstpage}{%
  \rhead{
	Elliott Daimler \\
	MATH 4200 \\
	Dr. Gay \\ 
	MWF 9:10am \\
	\date\today
  }
  
}


\begin{document}
\maketitle %This command prints the title based on information entered above

\thispagestyle{firstpage}
%Section and subsection automatically number unless you put the asterisk next to them.
\section*{Exercise 2.4}

Claim: The "standard topology" $\mathcal{T}_{std}$ on $\mathbb{R}^n$ satisfies the four conditions of a topology.

\begin{proof}
  It is vacuously true that $\emptyset$ is open.  Next, let $\epsilon = 1$; then 
  for every point $p \in \mathbb{R}$, $B(p, \epsilon) \subset \mathbb{R}$, so $\mathbb{R}$ is open. \\ 
  \\ 
  For any $U$, $V \in \mathcal{T}_{std}$, if $U \cap V = \emptyset$, then $U \cap V$ is open.  
  If $U \cap V \neq \emptyset$, then there is a point $p \in U \cap V$.  Let 
  $B_U(p, \epsilon_U)$ be the open ball in $U$ around center $p$, and let 
  $B_V(p, \epsilon_V)$ be the open ball in $V$ around center $p$.  Since 
  $B_U$ and $B_V$ are concentric, at least one of the following is true: \\
  \\ 
  Case 1: $B_U \subset B_V$.  Then $B_U \subset U$ and $B_U \subset V$.  \\ 
  \\
  Case 2: $B_V \subset B_U$.  Then $B_V \subset V$ and $B_V \subset U$. \\ 
  \\ 
  So for every $U$, $V \in \mathcal{T}_{std}$, $U \cap V \in \mathcal{T}_{std}.$\\ 
  \\
  Last, let $\lambda$ be an infinite subset of $\mathcal{T}_{std}$; then for 
  every point $p \in \cup_{\alpha \in \lambda} U_{\alpha}$ there is a $U_{\beta} \in \lambda$ 
  where $p \in U_{\beta}$.  It follows that there is an open ball $B(p, \epsilon) \subset U_{\beta}$ 
  for $\epsilon > 0$, so $B(p, \epsilon) \subset \cup_{\alpha \in \lambda} U_{\alpha}$ and 
  $\cup_{\alpha \in \lambda} U_{\alpha} \in \mathcal{T}_{std}$.
\end{proof}


\section*{Exercise 2.11}

\subsection*{1}

For the standard topology on $\mathbb{R}$ with $a, b \in \mathbb{R}$, a closed interval $A = [a, b ]$ has limit points 
$a$ and $b$, because any open interval containing $a$ or $b$ also contains other elements of $A$.  The elements $a, b$ are also 
contained in $A$.\\ 
\\ 
For the finite complement topology on $\mathbb{R}$, the open set $A = (-\infty, 0 ) \cup ( 0, \infty )$ has a limit point $p = 1$; 
this is because for any open set $B \subset \mathbb{R}$ where $p \in B$, $B$ has infinite points and therefore must contain other points in $A$.

\newpage 

\subsection*{2}
For the standard topology on $\mathbb{R}$ with $a, b \in \mathbb{R}$, an open interval $A = (a, b)$ has limit points $a$ and $b$ because any 
open set containing these two points also contains other points in $A$, but $a$, $b$ are not contained in $A$.\\ 
\\ 
For the finite complement topology on $\mathbb{R}$, the open set $A = (-\infty, 0) \cup (0, \infty)$ has a limit point $p = 0$.  It is 
a limit point because for any open set $B \subset \mathbb{R}$ where $p \in B$, $B$ has infinite points so it must contain other points in $A$.
 
\subsection*{3} 
Any set $A$ in the topology on $\mathbb{R}$ has no limit points, because for any point $p$ in $\mathbb{R}$, the set $\{ p \}$ is open, so $\{p\} - \{p\} \cap A = 0$. 
It follows that any point in $A$ is an isolated point.  \\ 
\\ 
A closed set $A$ in the finite complement topology on $\mathbb{R}$ is finite, so for any point $p \in A$, there is an open set $B$ containing 
$p$ for which $B - \{p\} \cap A = \emptyset$.  It follows that $A$ has no limit points 
and $p$ is an isolated point of $A$.

\subsection*{4}
As explained in part 3), a closed set $A$ in the finite complement topology on $\mathbb{R}$ has no limit points, 
so for any point $p \notin A$, $p$ is not a limit point of $A$.\\ 
\\ 
In the set of integers $\mathbb{Z}$ with the standard topology on $\mathbb{R}$, 
any non-integer rational $p \in (\mathbb{Q} - \mathbb{Z})$ is not contained 
in $\mathbb{Z}$; neither is it a limit point of $\mathbb{Z}$, because 
for any non-integer rational $p$, there is an integer $n$ such that 
$n < p < n+1$, so there exists an open interval $(n, n+1)$ containing $p$ 
where $(n, n+1) - \{p\} \cap \mathbb{Z} = \emptyset$.  


\section*{Exercise 2.12}
\subsection*{1}
Every set with the discrete topology is closed, because every set 
$B \subset X$ is open, therefore $X - B$ is closed.

\subsection*{2}
Both sets $X$, $\emptyset$ in the indiscrete topology $\{X, \emptyset \}$ are closed, 
because both sets are open, so the complements $X - \emptyset = X$ and 
$X - X = \emptyset$ are closed.  

\subsection*{3} 
A set $A \subset X$ is open in the finite complement topology when its complement $X - A$ 
is finite, so every finite set in this topology is closed, including the 
empty set $\emptyset$.  The entire set $X$ is also closed, because the 
empty set is open, and its complement $X - \emptyset = X$ is closed.

\subsection*{4}
A set $A \subset X$ is open in the countable complement topology on $X$ when its complement 
is countable, so every countable set in $X$ is closed, including the empty set 
$\emptyset$.  The entire set $X$ is also closed, because the empty set is open, 
and its complement $X - \emptyset = X$ is closed. 


\section*{Exercise 2.18}
\subsection*{1}
In the standard topology $\mathcal{T}_{std}$ on $\mathbb{R}$, the finite 
union $\cup_{i=1}^n [a_i, b_i]$ of any finite set of closed intervals $[a_i, b_i] \in \mathbb{R}$ 
is closed, but not open.\\ 
\\ 
Any finite, nonempty subset $A \subset \mathbb{R}$ with the finite complement topology on $\mathbb{R}$ 
is closed, but not open.

\subsection*{2}
For any finite, nonempty set $A \subset \mathbb{R}$ with the finite complement 
topology on $\mathbb{R}$, its complement $\mathbb{R} - A$ is open, but not closed.\\ 
\\ 
Let $\lambda$ be a collection of open intervals $(a, b) \in \mathbb{R}$ with the standard 
topology on $\mathbb{R}$; then if the union $A= \cup_{(a, b) \in \lambda} (a, b)$ is open
if $A \subsetneq \mathbb{R}$. 

\subsection*{3} 
Both elements $X, \emptyset$ in the indiscrete topology $\{X, \emptyset\}$ 
on $X$ are open and closed. \\ 
\\ 
Every set $A \subset \mathbb{R}$ with the discrete topology on $\mathbb{R}$ is 
both open and closed.

\subsection*{4}
In the standard topology on $\mathbb{R}$ with $a, b \in \mathbb{R}$ with $a \neq b$, intervals $(a, b]$ 
that are left-open, right-closed, and intervals $[a, b)$ that are right-open, left-closed, 
are neither open nor closed.\\ 
\\ 
Nonempty proper subsets $A \subsetneq X$ of a set $X$ with the indiscrete topology 
$\{X, \emptyset \}$ are neither open nor closed.

\newpage 

\section*{Exercise 2.21}
For $a, b, c \in \mathbb{R}$ with $a < b < c$, let $A = (a, b)$, $B = (a, b) \cup \{c\}$, 
and let $\mathbb{Q}$ be the rational numbers.

\subsection*{1}
In the discrete topology on $\mathbb{R}$, every set is open and closed, so 
$\overline{A} = A$, $\overline{B} = B$, and $\overline{C} = C$.

\subsection*{2}
For any nonempty subset $U$ of $X$ in the indiscrete topology $\{X, \emptyset \}$, 
every point in $X$ is a limit point of $U$.  So the closures of $A$, $B$, and $\mathbb{Q}$ 
are all $X$.

\subsection*{3} 
The sets $A$, $B$, and $C$ are infinite sets with infinite complements in $\mathbb{R}$, 
so they are neither open nor closed in the finite complement topology.  
Any open set $U$ in this topology will have a finite complement, so $U$ must 
intersect $A$, $B$, and $C$ at infinitely many points.  It follows that 
any point in $\mathbb{R}$ is a limit point of $A$, $B$, and $C$, so the closure 
of each of these sets is $\mathbb{R}$. 

\subsection*{4}
In the standard topology on $\mathbb{R}$, the limit points of $A$ include all points in $(a, b)$ 
and the points $a$, $b$, so the closure of $A$ is $[a, b]$.\\ 
\\ 
Similarly, the limit points of $B$ include all points in $(a, b)$, and the points $a$, $b$.  
The point $c$ is not a limit point of $B$, so the closure of $B$ is $[a, b] \cup \{c\}$.\\ 
\\ 
Any nonempty open subset in $\mathbb{R}$ has a nonempty intersection with $\mathbb{Q}$, 
so every point in $\mathbb{R}$ is a limit point of $\mathbb{Q}$, and the closure of $\mathbb{Q}$ 
is $\mathbb{R}$.

\section*{Exercise 2.29}
As in Exercise 2.21, we will use sets $A = (a, b)$, $B = (a, b) \cup \{c\}$ for 
$a$, $b$, $c \in \mathbb{R}$ and $a < b < c$, and the rational numbers $\mathbb{Q}$.

\subsection*{1}
In the discrete topology on $\mathbb{R}$, every subset of $\mathbb{R}$ is open, 
so the interiors of $A$, $B$, and $\mathbb{Q}$ are $A$, $B$, and $\mathbb{Q}$, 
respectively.\\ 
\\ 
Every subset of $\mathbb{R}$ is also closed in this topology, so each subset 
along with its complement are closed.  It follows that the boundaries of $A$, 
$B$, and $\mathbb{Q}$ are all empty.

\subsection*{2}
The only open subset of $A$, $B$, and $C$ in the indiscrete topology $\{\mathbb{R}, \emptyset \}$ is the empty set $\emptyset$, 
so the interior of each of these sets is empty.  The only closed set containing each 
of these sets is the entire set $\mathbb{R}$, so $\overline{A} = \mathbb{R}$, $\overline{B} = \mathbb{R}$, and $\overline{\mathbb{Q}} = \mathbb{R}$.  
Likewise, the closure of each of their complements is also $\mathbb{R}$, so $\overline{A} \cap \overline{\mathbb{R}-A} = \mathbb{R}$, 
$\overline{B} \cap \overline{\mathbb{R}-B} = \mathbb{R}$, and $\overline{\mathbb{Q}} \cap \overline{\mathbb{R}-\mathbb{Q}} = \mathbb{R}$.  These 
are the boundaries of each set.

\subsection*{3} 
The only open set contained in $A$ in the finite complement topology on $\mathbb{R}$ is the 
empty set $\emptyset$, and the same is true for sets $B$ and $\mathbb{Q}$, so the interior 
of each of these sets is empty.  Every point in this topology is a limit point, so 
$\overline{A} = \mathbb{R}$, $\overline{B} = \mathbb{R}$, and $\overline{\mathbb{Q}} = \mathbb{R}$.  The complement of each of these 
sets in $\mathbb{R}$ is also infinite, so $\overline{\mathbb{R}-A} = \mathbb{R}$, $\overline{\mathbb{R}-B} = \mathbb{R}$, 
and $\overline{\mathbb{R}-\mathbb{Q}} = \mathbb{R}$.  It follows that $\overline{A} \cap \overline{\mathbb{R}-A} = \mathbb{R}$, 
$\overline{B} \cap \overline{\mathbb{R}-B} = \mathbb{R}$, and $\overline{\mathbb{Q}} \cap \overline{\mathbb{R}-\mathbb{Q}} = \mathbb{R}$.  
These are the boundaries of each set.

\subsection*{4}
In the standard topology on $\mathbb{R}$, $A$ is open, so $\mathrm{Int}(A) = A$.  
The interval $(a, b)$ is open, and the only open set contained in $\{c\}$ is 
the empty set $\emptyset$, so $\mathrm{Int}(B) = (a, b)$.  The only open set contained 
in $\mathbb{Q}$ is the empty set $\emptyset$, so $\mathrm{Int}(\mathbb{Q}) = \emptyset$. \\ 
\\ 
The closure of $A$ in the standard topology on $\mathbb{R}$ is $[a, b]$, and the closure of $\mathbb{R} - A$ is $(-\infty, a] \cup [b, \infty)$, 
so $\overline{A} \cap \overline{\mathbb{R}-A} = \{a, b\}$ is the boundary.  Similarly, the closure of $B$ is $[a, b] \cup \{c\}$ and 
the closure of $\mathbb{R} - B$ is $(-\infty, a] \cup [b, \infty) \cup \{c\}$, so 
$\overline{B} \cap \overline{\mathbb{R}-B} = \{a, b, c\}$ is the boundary.  The closure of 
$\mathbb{Q}$ is $\mathbb{R}$, and the closure of $\mathbb{R} - \mathbb{Q}$ is 
$\mathbb{R}$, so the boundary of $\mathbb{Q}$ is $\overline{\mathbb{Q}} \cap \overline{\mathbb{R} - \mathbb{Q}} = \mathbb{R}$. 


\end{document}