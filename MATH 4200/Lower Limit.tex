%This is my super simple Real Analysis Homework template

\documentclass{article}
\usepackage[utf8]{inputenc}
\usepackage[english]{babel}
\usepackage[]{amsthm} %lets us use \begin{proof}
\usepackage[]{amssymb} %gives us the character \varnothing
\usepackage{fancyhdr}
\usepackage{amsmath}
\usepackage[margin=1in]{geometry} %reduce margins to 0.5 inch

\setlength{\parindent}{0pt}
\setlength{\parskip}{1em}

%\pagestyle{fancy}
\makeatletter
\renewcommand*\env@matrix[1][*\c@MaxMatrixCols c]{%
  \hskip -\arraycolsep
  \let\@ifnextchar\new@ifnextchar
  \array{#1}}
\makeatother

\newcommand{\Tau}{\mathrm{T}}

\renewcommand*\descriptionlabel[1]{\hspace\labelsep\normalfont #1}

\title{ }
\date{}
%This information doesn't actually show up on your document unless you use the maketitle command below

\fancypagestyle{firstpage}{%
  \rhead{
    Elliott Daimler \\
    MATH 4200 \\
	\date\today
  }
  
}


\begin{document}
\maketitle %This command prints the title based on information entered above

\thispagestyle{firstpage}
%Section and subsection automatically number unless you put the asterisk next to them.
\section*{Exercise 4.5}

Claim: The set of rationals $\mathbb{R}$ with the lower limit topology is normal.

\begin{proof}
    The approach used below employs the fact that for $x$, $y \in \mathbb{R}_{LL}$ where $x < y$, there exist disjoint open 
    sets $[x, y)$ and $[y, z)$ for some $z > y$.  \\ 
    \\ 
    Let $A$, $B$ be disjoint closed sets 
    in $\mathbb{R}_{LL}$, and let $a$ be any point in $A$.  Then for any 
    points $b_i \in B$ where $a < b_i$, there is some minimum point $\mathrm{Min}(b_i) = b_{min} \in B$ 
    where $a < b_{min}$.  It follows that there 
    is some open set $U_a = [a, b_{min})$ containing $a$ that contains no points 
    in $B$, and the union of open sets 

    \begin{alignat*}{1}
        \bigcup_{a \in A} U_a \supset A
    \end{alignat*}
    is open and contains no points in $B$.  Repeating this approach for open sets 
    $V_b = [b, a_{min})$ containing points $b$ in $B$, we have the union of opens sets
    \begin{alignat*}{1}
        \bigcup_{b \in B} V_b \supset B
    \end{alignat*}

    which is open, and the unions are disjoint:

    \begin{alignat*}{1}
        \biggl(\bigcup_{a \in A} U_a \biggr) \bigcap \biggl(\bigcup_{b \in B} V_b \biggr) = \emptyset.
    \end{alignat*}
\end{proof}





\end{document}