%This is my super simple Real Analysis Homework template

\documentclass{article}
\usepackage[utf8]{inputenc}
\usepackage[english]{babel}
\usepackage[]{amsthm} %lets us use \begin{proof}
\usepackage[]{amssymb} %gives us the character \varnothing
\usepackage{fancyhdr}
\usepackage{amsmath}
\usepackage[margin=1in]{geometry} %reduce margins to 0.5 inch

\setlength{\parindent}{0pt}
\setlength{\parskip}{1em}

%\pagestyle{fancy}
\makeatletter
\renewcommand*\env@matrix[1][*\c@MaxMatrixCols c]{%
  \hskip -\arraycolsep
  \let\@ifnextchar\new@ifnextchar
  \array{#1}}
\makeatother

\newcommand{\Tau}{\mathrm{T}}

\renewcommand*\descriptionlabel[1]{\hspace\labelsep\normalfont #1}

\title{\textbf{Section 3.4 Solutions}}
\date{}
%This information doesn't actually show up on your document unless you use the maketitle command below

\fancypagestyle{firstpage}{%
  \rhead{
	Elliott Daimler \\
	MATH 4200 \\
	Dr. Gay \\ 
	MWF 9:10am \\
	\date\today
  }
  
}


\begin{document}
\maketitle %This command prints the title based on information entered above

\thispagestyle{firstpage}
%Section and subsection automatically number unless you put the asterisk next to them.
\section*{Exercise 3.30}

Theorem: Let $(X, \mathcal{T})$ be a topological space, and let $(Y, \mathcal{T}_Y)$ be a subspace.  If $\mathcal{B}$ is a 
basis for $\mathcal{T}$, then $\mathcal{B}_Y = \{B \cap Y | B \in \mathcal{B} \}$ is a basis for $\mathcal{T}_Y$.

\begin{proof}
  The set $\mathcal{B}$ is a basis for $\mathcal{T}$, so for any $x \in X$ there is a $B \in \mathcal{B}$ where $x \in B$.  Since $Y$ is a subset of 
  $X$, for any $y \in X$ there is also a $C$ where $y \in C$, so $y \in Y \cap C \in \mathcal{B}_Y$. \\ 
  \\ 
  Next, let $U_Y$, $V_Y \in \mathcal{B}_Y$ be given where $U_Y \cap V_Y \neq \emptyset$, and let $p \in U_Y \cap V_Y$ be given.  There are some $U \in \mathcal{B}$ 
  and $V \in \mathcal{B}$ such that $U \cap Y = U_Y$, $V \cap Y = V_Y$, so $p \in U \cap V$.  It follows that there is some $W \subset (U \cap V)$ where $p \in W$, 
  and $(W \cap Y) \subset (U_Y \cap V_Y)$, so $p \in (W \cap Y)$.
\end{proof}

\section*{Exercise 3.31} 

\subsection*{1} 
$D = \left\lbrace \left( x, \frac{1}{2}\right) | 0 < x < 1 \right\rbrace$. 


\subsection*{2} 
$E = \left\lbrace \left( \frac{1}{2}, y\right) | 0 < y < 1 \right\rbrace$. 



\subsection*{3} 
$F = \left\lbrace \left( x, 1 \right) | 0 < x < 1 \right\rbrace$. 



\end{document}