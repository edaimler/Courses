%This is my super simple Real Analysis Homework template

\documentclass{article}
\usepackage[utf8]{inputenc}
\usepackage[english]{babel}
\usepackage[]{amsthm} %lets us use \begin{proof}
\usepackage[]{amssymb} %gives us the character \varnothing
\usepackage{fancyhdr}
\usepackage{amsmath}
\usepackage[margin=1in]{geometry} %reduce margins to 0.5 inch

\setlength{\parindent}{0pt}
\setlength{\parskip}{1em}

%\pagestyle{fancy}
\makeatletter
\renewcommand*\env@matrix[1][*\c@MaxMatrixCols c]{%
  \hskip -\arraycolsep
  \let\@ifnextchar\new@ifnextchar
  \array{#1}}
\makeatother

\newcommand{\Tau}{\mathrm{T}}

\renewcommand*\descriptionlabel[1]{\hspace\labelsep\normalfont #1}

\title{\textbf{Section 3.4 Solutions}}
\date{}
%This information doesn't actually show up on your document unless you use the maketitle command below

\fancypagestyle{firstpage}{%
  \rhead{
	Elliott Daimler \\
	MATH 4200 \\
	Dr. Gay \\ 
	MWF 9:10am \\
	\date\today+
  }
  
}


\begin{document}
\maketitle %This command prints the title based on information entered above

\thispagestyle{firstpage}
%Section and subsection automatically number unless you put the asterisk next to them.
\section*{Theorem 3.30}

Theorem: Let $(X, \mathcal{T})$ be a topological space, and let $(Y, \mathcal{T}_Y)$ be a subspace.  If $\mathcal{B}$ is a 
basis for $\mathcal{T}$, then $\mathcal{B}_Y = \{B \cap Y | B \in \mathcal{B} \}$ is a basis for $\mathcal{T}_Y$.

\begin{proof}
  The set $\mathcal{B}$ is a basis for $\mathcal{T}$, so for any $x \in X$ there is a $B \in \mathcal{B}$ where $x \in B$.  Since $Y$ is a subset of 
  $X$, for any $y \in X$ there is also a $C$ where $y \in C$, so $y \in Y \cap C \in \mathcal{B}_Y$. \\ 
  \\ 
  Next, let $U_Y$, $V_Y \in \mathcal{B}_Y$ be given where $U_Y \cap V_Y \neq \emptyset$, and let $p \in U_Y \cap V_Y$ be given.  There are some $U \in \mathcal{B}$ 
  and $V \in \mathcal{B}$ such that $U \cap Y = U_Y$, $V \cap Y = V_Y$, so $p \in U \cap V$.  It follows that there is some $W \subset (U \cap V)$ where $p \in W$, 
  and $(W \cap Y) \subset (U_Y \cap V_Y)$, so $p \in (W \cap Y)$.
\end{proof}

\section*{Exercise 3.31} 



\subsection*{1} 

$D = \left\lbrace \left( x, \frac{1}{2}\right) | 0 < x < 1 \right\rbrace$. \\ 
\\ 
For any point $(x, y)$ in the lexographically ordered square, we have the open set 
 $U = \{ (w, z) | (x, y) < (w, z) \}$.  For any point $(w, z)$ in the square, if $x < w$, then 
 $(w, z) \in U$, so the interval $(x, 1) \times \{\frac{1}{2}\} \subset (U \cap D)$ is open in $D$.  If $x = w$ and $\frac{1}{2} < z$, then 
 $(w, z)$ is also in $U$, so the interval $[x, 1) \times \{\frac{1}{2}\} \subset (U \cap D)$ is open in $D$, and the point $\{x \} \times \{\frac{1}{2}\}$ 
 is open in $D$.\\ 
 \\ 
The set $V = \{ (w, z) | (w, z) < (x, y) \}$ in the square contains all points $(w, z)$ where $w < x$, so  
$(0, x) \times \{ \frac{1}{2} \} \subset (V \cap D)$.  For any $w = x$, $z < y$, the point $(w, z)$ is also contained in $V$, 
so $(0, x] \times \{ \frac{1}{2} \} \subset (V \cap D)$ is open in $D$. \\ 
\\ 
Given point $(x, y)$ and set $W = \{ (w, z) | (w, z) < (x, y) \} \cup \{ (a, b) | (x, y) < (a, b) \}$ in the square, the subset 
$(w, a) \times \{ \frac{1}{2} \} \subset (W \cap D)$ is open in $D$.\\ 
\\ 
To summarize, for any point $(x, \frac{1}{2}) \in D$, the sets $(x, 1) \times \{ \frac{1}{2} \}$ and 
$[x, 1) \times \{ \frac{1}{2} \}$ are open in $D$; $(w, a) \times \{ \frac{1}{2} \}$ is open, where $w < x < a$; 
and and the point $\{x\} \times \{\frac{1}{2}\}$ is open in $D$, so any set of the form $[w, x] \times \{\frac{1}{2}\}$ for $w < x$ 
or of the form $[x, w]\times \{\frac{1}{2}\}$ for $x < w$ are open in $F$.. \\

\newpage

\subsection*{2} 
$E = \left\lbrace \left( \frac{1}{2}, y\right) | 0 < y < 1 \right\rbrace$. \\
\\
For any point $(x, y)$ in the square, we have the open set 
 $U = \{ (w, z | (x, y) < (w, z) \}$.  For any point $(w, z)$ in the square, if $x < w$, then 
 $(w, z) \in U$, so for points where $x < \frac{1}{2}$ in the square, the set $\{ \frac{1}{2} \} \times [y, 1) \subset (U \cap E)$ is open in $E$.  
 Similarly, the point $\{\frac{1}{2}\} \times \{y\}$ is open in $E$.  
 If $x = \frac{1}{2}$ and $y < z$, then the set $\{ \frac{1}{2} \} \times (y, 1) \subset (U \cap E)$ is open in $E$. \\ 
 \\ 
The set $V = \{ (w, z) | (w, z) < (x, y) \}$ in the square contains all points $(w, z)$ where $w < x$, so for points where $ \frac{1}{2} < x$, the set 
$\{ \frac{1}{2} \} \times (0, y] \subset (V \cap E)$.  Where $\frac{1}{2} = x$, $z < y$, the point $(w, z)$ is also contained in $V$, 
so $\frac{1}{2} \times (0, y) \subset (V \cap E)$ is open in $E$. \\ 
\\ 
Given point $(x, y)$ and set $W = \{ (w, z) | (w, z) < (x, y) \} \cup \{ (a, b) | (x, y) < (a, b) \}$ in the square, the subset 
$\{ \frac{1}{2} \} \times (z, b) \subset (W \cap E)$ is open in $E$.\\ 
\\ 
To summarize, for any point $(\frac{1}{2}, y) \in E$, the sets $\{ \frac{1}{2} \} \times [1, y)$ and 
$\{ \frac{1}{2} \} \times (y, 1) $ are open in $E$, and $\{ \frac{1}{2} \} \times (z, b)$ is open, where $z < y < b$. 
The point $\{\frac{1}{2}\} \times \{y\}$ is open in $E$, so sets of the form $\{\frac{1}{2}\} \times [z, y]$ 
where $z < y$ or of the form $\{\frac{1}{2}\} \times [y, z]$ where $y < z$ are open in $E$.\\


\subsection*{3} 
$F = \left\lbrace \left( x, 1 \right) | 0 < x < 1 \right\rbrace$.  \\ 
\\ 
Given a point $(x, y)$ in the lexographically ordered square and the open set $U = \{ (w, z | (x, y) < (w, z) \}$, then if 
$x < w$, the set $(x, 1) \times \{1 \} \subset (U \cap F)$ is open in $F$.  If $x = w$ and $z < 1$, then the set 
$[x, 1) \times \{ 1 \} \subset (U \cap F)$ is open in $F$, and the point $\{x\} \times \{1\}$ is open in $F$.\\ 
\\ 
The open set $V$ in the square where $V = \{ (w, z) | (w, z) < (x, y) \}$ contains all points where $w < x$, so 
$(0, x) \times \{1 \} \subset (V \cap F)$ is open in $F$.  Points where $w = x$ and $z < 1$ are also in $V$, 
so $(0, x] \times \{1 \} \subset (V \cap F)$ is also open in $F$. \\ 
\\ 
Last, similar to the above topologies, the open set $W = \{ (w, z) | (w, z) < (x, y) \} \cup \{ (a, b) | (x, y) < (a, b) \}$ 
in the square contains all points $(x, y)$ where $w < x < a$ and $z < y < b$, so the set 
$(w, a) \times \{ 1 \} \subset (W \cap F)$ is open in $F$.\\ 
\\ 
The sets $[x, 1) \times \{ 1 \} \subset (U \cap F)$ and $(0, x] \times \{1 \} \subset (V \cap F)$ are open in $F$, 
and $(w, a) \times \{ 1 \} \subset (W \cap F)$ is open in $F$, where $w < x < a$.  Moreover, the point $\{x\} \times \{1\}$ is open in $F$, 
so any set of the form $[w, x] \times \{1\}$ for $w < x$ or of the form $[x, w]\times \{1\}$ for $x < w$ are open in $F$.\\
\\ 
The open sets of these topological subspaces include open intervals, half-open-and-half-closed intervals, closed intervals, 
and points.  Taking the union of these sets in each space, one can construct any subset of the respective space, so these 
topologies resemble the discrete topology.\\ 

\end{document}