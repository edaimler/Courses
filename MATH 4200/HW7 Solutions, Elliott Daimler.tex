%This is my super simple Real Analysis Homework template

\documentclass{article}
\usepackage[utf8]{inputenc}
\usepackage[english]{babel}
\usepackage[]{amsthm} %lets us use \begin{proof}
\usepackage[]{amssymb} %gives us the character \varnothing
\usepackage{fancyhdr}
\usepackage{amsmath}
\usepackage[margin=1in]{geometry} %reduce margins to 0.5 inch

\setlength{\parindent}{0pt}
\setlength{\parskip}{1em}

%\pagestyle{fancy}
\makeatletter
\renewcommand*\env@matrix[1][*\c@MaxMatrixCols c]{%
  \hskip -\arraycolsep
  \let\@ifnextchar\new@ifnextchar
  \array{#1}}
\makeatother

\newcommand{\Tau}{\mathrm{T}}

\renewcommand*\descriptionlabel[1]{\hspace\labelsep\normalfont #1}

\title{\textbf{Section 5.1 and Order Topology Solutions}}
\date{}
%This information doesn't actually show up on your document unless you use the maketitle command below

\fancypagestyle{firstpage}{%
  \rhead{
	Elliott Daimler \\
	MATH 4200 \\
	Dr. Gay \\ 
	MWF 9:10am \\
	\date\today
  }
  
}


\begin{document}
\maketitle %This command prints the title based on information entered above

\thispagestyle{firstpage}
%Section and subsection automatically number unless you put the asterisk next to them.
\section*{Theorem 5.5}

If $X$ and $Y$ are separable spaces, then $X \times Y$ is separable.

\begin{proof}
Both $X$ and $Y$ have countable dense subsets; call these subsets $A \subset X$ and $B \subset Y$.  
The product $A \times B$ is countable.  There is no closed proper subset of $X$ that contains $A$.  \\
\\
Let $p = (x, y) \in X \times Y$ be given, where $x \in A$ and $y \in B$.  Then for every open set 
$U \subset X$ where $U \ni x$, $(U - x) \cap A$ is nonempty.  Similarly, for every open set 
$V \subset Y$ where $V \ni y$, $(V - y) \cap B$ is nonempty.  It follows that $U \times V$ 
contains $p$, and $(U \times V - p) \cap A \times B$ is nonempty, so $A \times B$ is separable.


\end{proof}

\section*{Order Topology Theorem} 

The order topology is normal.

\begin{proof}
  Let $(X, \mathcal{T}_{order})$ be the topological space made by the order topology with set $X$, and let 
  $A$, $B$ be disjoint closed sets in $X$.  Let $C$ be the complement of $A \cup B$ in $X$, or 
  $C = X - (A \cup B)$.  Let $a \in A$ be given.\\ 
  \\ 
  For some $b_1 \in B$, there is a set $(a, b_1) = \{x | a < x < b_1 \quad for \quad x \in X \}$ where $(a, b_1) \cap A = \emptyset$  
  and $(a, b_1) \cap B = \emptyset$.  Note that this set may be empty.  The set $(a, b_1)$ is open.  
  Remove some point $c$ from this set, and we have two open intervals, either of which may be empty: $(a, c)$ and $(c, b_1)$.   
  For some other $b_2 \in B$, there is a set $(b_2, a) = \{x | b_2 < x < a \quad for \quad x \in X \}$ where $(b_2, a) \cap A = \emptyset$ 
  and $(b_2, a) \cap B = \emptyset$.  Again, this set may be empty.  Removing a point $e$ from this set and proceeding in a similar manner as above, 
  we have two open intervals $(b_2, e)$ and $(e, a)$, either of which may be empty.  Then taking the union $(e, a) \cup \{a\} \cup (a, c)$, we 
  have an open set $(e, c) = \{ x | e < x < c \quad for \quad x \in X\}$ that contains $a$ and contains no points in $B$.  
  Assuming without loss of generality that $(e, a)$ is empty, the union above becomes $(b_2, a) \cup \{a\} \cup (a, c)$, and 
  still contains no points in $B$ and is open. \\ 
  \\
  For $b_1$, we can construct an open set (possibly empty) $(b_1, f)$ using the method above and take a union $(c, b_1) \cup \{b\} (b_1, f)$ 
  which contains no points in $A$.  Proceeding in this manner for $d$ and all other elements of $A$ and $B$, we can construct an open set 
  containing every point in $A$ and every point in $B$, where each of these sets are disjoint.  Taking the union of all of these sets that 
  contain points in $A$, we have an open set containing all of $A$, and taking the union of all of these sets that contain points in $B$, 
  we have an open set containing all of $B$.  These open sets containing $A$ and $B$ are disjoint.
\end{proof}


\end{document}