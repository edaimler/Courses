%This is my super simple Real Analysis Homework template

\documentclass{article}
\usepackage[utf8]{inputenc}
\usepackage[english]{babel}
\usepackage[]{amsthm} %lets us use \begin{proof}
\usepackage[]{amssymb} %gives us the character \varnothing
\usepackage{fancyhdr}
\usepackage{amsmath}
\usepackage[margin=1in]{geometry} %reduce margins to 0.5 inch

\setlength{\parindent}{0pt}
\setlength{\parskip}{1em}

%\pagestyle{fancy}
\makeatletter
\renewcommand*\env@matrix[1][*\c@MaxMatrixCols c]{%
  \hskip -\arraycolsep
  \let\@ifnextchar\new@ifnextchar
  \array{#1}}
\makeatother

\newcommand{\Tau}{\mathrm{T}}

\renewcommand*\descriptionlabel[1]{\hspace\labelsep\normalfont #1}

\title{\textbf{Section 4.2 Solutions}}
\date{}
%This information doesn't actually show up on your document unless you use the maketitle command below

\fancypagestyle{firstpage}{%
  \rhead{
	Elliott Daimler \\
	MATH 4200 \\
	Dr. Gay \\ 
	MWF 9:10am \\
	\date\today
  }
  
}


\begin{document}
\maketitle %This command prints the title based on information entered above

\thispagestyle{firstpage}
%Section and subsection automatically number unless you put the asterisk next to them.
\section*{Theorem 4.16}

Let topological spaces $X$ and $Y$ be Hausdorff.  Then $X \times Y$ is Hausdorff. 

\begin{proof}
  Let $a = (a_x, a_y)$ and $b = (b_x, b_y)$ be points in $X \times Y$ where $a_x$, $b_x \in X$ and $a_y$, $b_y \in Y$.  If 
  $a \neq b$, then $a_x \neq b_x$ or $a_y \neq b_y$.  \\
  \\
  Suppose $a_x \neq b_x$; since $X$ is Hausdorff, there exist open sets $U_a \ni a_x$ and $U_b \ni b_x$ such that $U_a \cap U_b = \emptyset$, 
  so for any open sets $V_a$, $V_b \subset Y$ where $V_a \ni a_y$ and $V_b \ni b_y$, we have 

  \begin{alignat*}{1}
    (U_a \times V_a) \cap (U_b \times V_b) &= \left[ \pi^{-1}(U_a) \cap \pi^{-1}(V_a) \right] \cap \left[ \pi^{-1}(U_b) \cap \pi^{-1}(V_b) \right] \\ 
    &= \emptyset.
  \end{alignat*}
  If $a_y \neq b_y$, then by the same reasoning as above we have open sets $V_a \ni a_y$ and $V_b \ni b_y$ where $V_a \cap V_b = \emptyset$, 
  and sets $(U_a \times V_a)$ and $(U_b \times V_b)$ are disjoint, so $X \times Y$ is Hausdorff.
\end{proof}



\section*{Theorem 4.17} 

Let topological spaces $X$ and $Y$ be regular.  Then $X \times Y$ is regular.

\begin{proof}
  Let $p = (p_x, p_y)$ be a point in $X \times Y$, and let $U_x \times U_y$ be an open set in $X \times Y$ where 
  $U_x$ is a basic element in $X$ and $U_y$ is a basic element in $Y$.  Then $p_x \in U_x$ and $p_y \in U_y$. $X$ and $Y$ are each regular, so by Theorem 4.8, 
  there exist open sets $V_x \ni p_x$, $V_y \ni p_y$ where $V_x \subset U_x$ and $V_y \subset U_y$ such that 
  $\overline{V_x} \subset U_x$ and $\overline{V_y} \subset U_y$.  The point $p$ is in $V_x \times V_y$, and 
  $\overline{V_x} \times \overline{V_y} \subset U_x \times U_y$.  \\ 
  \\ 
  It remains to show that $\overline{V_x} \times \overline{V_y}$ is closed.  Let $q = (q_x, q_y)$ be a limit point of $V_x \times V_y$, 
  and let $Q_x \times Q_y \ni q$ be open where $Q_x \subset X$ and $Q_y \subset Y$ are basic elements.  Then $q_x \in Q_x$ and $q_y \in Q_y$, 
  and $(Q_x - q_x) \cap V_x$ is nonempty, and $(Q_y - q_y) \cap V_y$ is nonempty, so $q_x \in \overline{V_x}$ and $q_y \in \overline{V_y}$.  It follows that 
  $q \in \overline{V_x} \times \overline{V_y}$ and $\overline{V_x} \times \overline{V_y}$ is closed.  By Theorem 4.8, $X \times Y$ is regular.
\end{proof}

\newpage 

\section*{Exercise 4.18} 

Let the set $A \in \mathbb{R}_{LL} \times \mathbb{R}_{LL}$ be the function $f(x) = y = 1 - x$ from $x \in \left[ 0, 1 \right]$, and 
$g(x) = y = -1 + x$ from $x \in \left[ 0, 1 \right]$.  Let $U \subset \mathbb{R}_{LL} \times \mathbb{R}_{LL}$ be open where 
$U = [0, 2) \times [-1, 2)$; then $U$ contains $A$.  It is not true that $ 


\end{document}