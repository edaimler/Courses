%This is my super simple Real Analysis Homework template

\documentclass{article}
\usepackage[utf8]{inputenc}
\usepackage[english]{babel}
\usepackage[]{amsthm} %lets us use \begin{proof}
\usepackage[]{amssymb} %gives us the character \varnothing
\usepackage{fancyhdr}
\usepackage{amsmath}
\usepackage[margin=1in]{geometry} %reduce margins to 0.5 inch

\setlength{\parindent}{0pt}
\setlength{\parskip}{1em}

%\pagestyle{fancy}
\makeatletter
\renewcommand*\env@matrix[1][*\c@MaxMatrixCols c]{%
  \hskip -\arraycolsep
  \let\@ifnextchar\new@ifnextchar
  \array{#1}}
\makeatother

\newcommand{\Tau}{\mathrm{T}}

\renewcommand*\descriptionlabel[1]{\hspace\labelsep\normalfont #1}

\title{\textbf{Section 6.3 Solutions}}
\date{}
%This information doesn't actually show up on your document unless you use the maketitle command below

\fancypagestyle{firstpage}{%
  \rhead{
	Elliott Daimler \\
	MATH 4200 \\
	Dr. Gay \\ 
	MWF 9:10am \\
	\date\today
  }
  
}


\begin{document}
\maketitle %This command prints the title based on information entered above

\thispagestyle{firstpage}
%Section and subsection automatically number unless you put the asterisk next to them.
\section*{Exercise 6.22}

Given the space $2^X$ and preimage $\pi^{-1}_x(\delta)$, where $\delta \in \{0, 1\}$ and 
$\pi^{-1}_x(\delta) = \{f \in \{0, 1\}^X | f(x) = \delta \}$, we have the subbasis $\mathcal{S} = \{\pi^{-1}_x(\delta) | x \in X, \delta \in \{0,1\} \}$.  
For any open cover $\mathcal{B} \subset \mathcal{S}$, let $B \in \mathcal{B}$ be given; then $B$ is of the form $\pi^{-1}_x(\delta)$ for some $x \in X$.  
Let $\gamma \in \{0,1\}$ such that $\gamma \neq \delta$; for any $y \notin B$, $y \in \pi^{-1}_x(\gamma)$.  \\ 
\\ 
It remains to show that for some $\pi^{-1}_x(0)$ in $\mathcal{B}$, $\pi^{-1}_x(1)$ is also in $\mathcal{B}$.  Suppose for the 
purpose of contradiction there is no $\pi^{-1}_x(0)$ in $\mathcal{B}$ where $\pi^{-1}_x(1)$ is also in $\mathcal{B}$;  
then $\mathcal{B} = \{ \pi^{-1}_x(\delta) | x \in X \quad and \quad \delta \in \{0, 1\} - \{\gamma \} \}$ and 

\begin{alignat*}{1}
    \bigcap _{x \in X, \gamma \neq \delta} \pi^{-1}_x(\gamma)
\end{alignat*}

is not in any subset of $\mathcal{B}$, but $\mathcal{B}$ covers $2^X$, so we have a 
contradiction.  So for any cover $\mathcal{B}$ of $2^X$, there is an $x \in X$ such that 
$\pi^{-1}_x(0) \in \mathcal{B}$ and $\pi^{-1}_x(1) \in \mathcal{B}$, making a finite subcover 
of two subbasic sets.

\newpage 


\section*{Exercise 6.24}

Let $U = [0, 2/3)$ and let $V = (1/3, 1]$.  Then for countably infinite set $\omega$, we have a cover 
$\mathcal{C} = \{U_i, V_i | i \in \omega \}$ of $[0, 1]^\omega$ with the 
box topology.  Each $C \in \mathcal{C}$ is of the form 

\begin{alignat*}{1}
    C &= \prod _{C_i \in \{U_i, V_i\}}^\omega C_i \\ 
    &= \bigg(  \prod _{C_i \in \{U_i, V_i\}, i \neq j}^\omega C_i \bigg) \times C_j .
\end{alignat*}

For any $j \in \omega$ and any $D_j \in \{U_j, V_j\}$ where $D_j \neq C_j$ then, we have 

\begin{alignat*}{1}
    D &= \bigg(  \prod _{C_i \in \{U_i, V_i\}, i \neq j}^\omega C_i \bigg) \times D_j \\ 
    &\neq \bigg(  \prod _{C_i \in \{U_i, V_i\}, i \neq j}^\omega C_i \bigg) \times C_j.
\end{alignat*}

Suppose for the purpose of contradiction that $D$ is not in $\mathcal{C}$.  Then $\mathcal{C}$ 
is not a cover of $[0, 1]^\omega$, a contradiction.  So $D$ must be in $\mathcal{C}$ for 
every $D_j \neq C_j$, and there are infinitely many necessary products for $\mathcal{C}$ 
to be a cover, so $\mathcal{C}$ has no finite subcover.

\end{document}