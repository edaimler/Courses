%This is my super simple Real Analysis Homework template

\documentclass{article}
\usepackage[utf8]{inputenc}
\usepackage[english]{babel}
\usepackage[]{amsthm} %lets us use \begin{proof}
\usepackage[]{amssymb} %gives us the character \varnothing
\usepackage{fancyhdr}
\usepackage{amsmath}
\usepackage[margin=1in]{geometry} %reduce margins to 0.5 inch

\setlength{\parindent}{0pt}
\setlength{\parskip}{1em}

%\pagestyle{fancy}
\makeatletter
\renewcommand*\env@matrix[1][*\c@MaxMatrixCols c]{%
  \hskip -\arraycolsep
  \let\@ifnextchar\new@ifnextchar
  \array{#1}}
\makeatother

\newcommand{\Tau}{\mathrm{T}}

\renewcommand*\descriptionlabel[1]{\hspace\labelsep\normalfont #1}

\title{\textbf{Section 3.5 Solutions}}
\date{}
%This information doesn't actually show up on your document unless you use the maketitle command below

\fancypagestyle{firstpage}{%
  \rhead{
	Elliott Daimler \\
	MATH 4200 \\
	Dr. Gay \\ 
	MWF 9:10am \\
	\date\today
  }
  
}


\begin{document}
\maketitle %This command prints the title based on information entered above

\thispagestyle{firstpage}
%Section and subsection automatically number unless you put the asterisk next to them.
\section*{Exercise 3.39}

For rationals $q$ in $\mathbb{Q}$, the set $Z$ consisting of all elements of $2^{\mathbb{R}}$ that are 0 on every rational coordinate 
and 0 or 1 on any irrational coordinate can be written as 

\begin{alignat*}{1}
  Z = \bigcap_{q \in \mathbb{Q}} \pi^{-1}_q(0).
\end{alignat*}

Any open set in the product space $2^{\mathbb{R}}$ is a finite intersection of preimages $\pi^{-1}_x(\delta)$ for $x$ in $\mathbb{R}$, and  
open sets containing some point $p$ in $Z$ must contain an element that is 0 for every rational number and 0 or 1 for every irrational.  
Any set in $2^{\mathbb{R}}$ that does not contain multiple points in $Z$ must have the form 

\begin{alignat*}{1}
  \bigcap_{q \in \mathbb{Q}} \pi^{-1}_q(1), 
\end{alignat*}

which is an infinite intersection of subbasis elements, and is not an open set.  So every set in $2^{\mathbb{R}}$ 
contains multiple points in $Z$, and the closure of $Z$ is the entire space $2^{\mathbb{R}}$.

\section*{Exercise 3.42} 

The set $2^{\mathbb{N}}$ with the box topology has basis $\prod_{\alpha \in \lambda}U_\alpha$ for each open set $U_\alpha$ in $X_\alpha$.  
Each $U_\alpha$ takes the form $U_\alpha(0)$ or $U_\alpha(1)$, where $U_\alpha(0) = \{ f \in \{0,1\}^{\mathbb{N}}| f(\alpha) = 0 \}$, 
and $U_\alpha(1) = \{ f \in \{0,1\}^{\mathbb{N}}| f(\alpha) = 1 \}$ for functions $f$ mapping each $\alpha$ to $0$ or $1$.  
Taking the product of open sets $U_\alpha$ for every $\alpha$ in $X$ produces a single set $\{\delta_\alpha\}_{\alpha \in \mathbb{N}}$ where each
$\delta = 0$ or $\delta = 1$.  This set is a point in in $2^{\mathbb{N}}$, and the point is a basis element in the box topology.  
Since every point in the topological space can be generated this way, the space is discrete. \\ 
\\ 
The set $2^{\mathbb{N}}$ with the product topology is generated by subbasis elements $\pi^{-1}_\alpha \{\delta\}$ where $\delta = 0$ 
or $\delta = 1$.  Each $\pi^{-1}_\alpha \{\delta\}$ contains every set $\{\delta_i | i \in \mathbb{N}\} \cup \{\delta_\alpha\}$, or 
all sets in $2^{\mathbb{N}}$ containing $\delta_\alpha$.  Each basis element in this topology is generated by the finite intersections 
of these preimages, $\cap_{i = 1}^n \pi^{-1}_i \{\delta\}$, so for each of these intersections, there are infinitely many preimages 
$\pi^{-1}_\alpha \{\delta\}$ that are not included in the intersection.  For some basis element $\cap_{i = 1}^n \pi^{-1}_i \{\delta\}$, 
let $C$ be the collection of preimages $\pi^{-1}_\alpha \{\delta\}$ that are not included in the intersection, or $\alpha \neq i$ 
for any $\alpha$, $i$.  Then the product $\pi^{-1}_\alpha \{\delta\} \times \cap_{i = 1}^n \pi^{-1}_i \{\delta\}$ contains all sets 
containing $\delta_\alpha$ and each value $\delta_i$.  Since there are infinitely many $\pi^{-1}_\alpha \{\delta\} \in C$, 
there are infinitely many such sets in each open set in the product topology, so there are no isolated points in $2^\mathbb{N}$ 
with the product topology.



\end{document}